\subsection{Recursos humanos}

Los costes relacionados con los recursos humanos hacen referencia a todo el gasto asociado al salario del personal que participa en el proyecto. 
En este caso, el proyecto es desarrollado por una única persona, el autor del mismo, pero el papel del autor se divide en varios roles o funciones, 
cada una de las cuales tiene un coste diferente asociado. Estos roles son los siguientes:
\begin{itemize}
    \item  \textbf{Product Owner:} Responsable de la planificación, seguimiento y control del proyecto. Salario anual: 55.000 €
    \item  \textbf{Ingeniero de datos:} Responsable de la obtención, transformación y carga de los datos. Salario anual: 32.000 €
    \item  \textbf{Desarrollador:} Responsable de tareas de desarrollo de software, incluyendo el desarrollo y pruebas de código. Salario anual: 29.000 €
\end{itemize}

Se han estimado un salarios por hora y el coste total para la empreasa de la siguiente manera(base reguladora + IRPF + Seguridad Social):

\begin{table}[H]
    \centering
    \begin{tabular}{|c|c|c|c|c|}
        \hline
        \textbf{Rol} & \textbf{Salario Bruto (hora)} & \textbf{Coste total para la empresa(hora)}\\
        \hline
        Product Owner (PO) & 26,44 & 20  € \\
        Ingeniero de datos (ID) & 15,38 & 15 € \\
        Desarrollador junior (DJ) & 13,94 & 18  € \\
        \hline
    \end{tabular}
    \caption{Resumen del tiempo aproximado para cada tarea. Elaboración propia.}
    \label{tab:estimaciones}
\end{table}

Las aproximaciones salariales obtenidas se han extraido basandose en los datos publicados por el portal \textit{Glassdoor} \cite{glassdoor}, 
asumiendo una media de 40 horas semanales y 14 pagas anuales. Así como un IRF del 30

\begin{table}[H]
    \centering
    \begin{tabular}{|c|c|c|c|c|}
        \hline
        \textbf{ID} & \textbf{Tarea} & \textbf{Tiempo} & \textbf{Roles} & \textbf{Roles}\\
        \hline
        \textbf{GP} & \textbf{Gestión del Proyecto} & \textbf{138 h} & \textbf{-}  & € \\
        \hline
        GP-1 & Contextualización y alcance & 20 h &  PO  & € \\
        GP-2 & Planificación temporal & 15 h &  PO  & € \\
        GP-3 & Gestión económica y sostenibilidad & 18 h &  PO  & € \\
        GP-4 & Entrega final de GEP & 10 h &  PO  & € \\
        GP-5 & Reuniones con el director y el departamento & 6 h &  PO,ID,DJ  & € \\
        GP-6 & Redacción de la memoria & 45 h & PO,ID,DJ & € \\
        GP-7 & Presentación de la defensa & 20 h &  PO  & € \\
        GP-8 & Defensa del proyecto & 4 h &  PO & € \\
        \hline
        \textbf{BI} & \textbf{Búsqueda e Investigación} & \textbf{50 h} & \textbf{-} & € \\
        \hline
        BI-1 & Estudio de las soluciones existentes & 15 h & ID,DJ & € \\
        BI-2 & Investigación de tecnologías y herramientas & 30 h & ID,DJ & € \\
        BI-3 & Estudio de soluciones alternativas & 5 h & ID,DJ & € \\
        \hline
        \textbf{DP} & \textbf{Desarrollo del proyecto} & \textbf{351 h} & \textbf{-} & € \\
        \hline
        DP-1 & Obtención de datos de almacenaje & 30 h & ID & € \\
        DP-2 & Modificación de los plugins de Slurm & 65 h & DJ & € \\
        DP-3 & Obtención de datos de cómputo & 45 h & ID & € \\
        DP-4 & Desarrollo del ETL & 120 h & ID & € \\
        DP-5 & Desarrollo del dashboard & 50 h & DJ & € \\
        DP-6 & Pruebas y validación & 25 h & DJ & € \\
        DP-7 & Despliegue en entorno productivo & 16 h & DJ & € \\
        \hline
    \end{tabular}
    \caption{Resumen del tiempo aproximado para cada tarea. Elaboración propia.}
    \label{tab:estimaciones}
\end{table}