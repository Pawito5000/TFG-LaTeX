\subsection{Contingencias}
Existe una gran posibilidad de que durante la ejecución del proyecto surjan imprevistos que 
requieran recursos adicionales no contemplados en el presupuesto inicial. Es por eso que 
para mitigar el riesgo que implican estos imprevistos, se ha decidido asignar un porcentaje 
del presupuesto total a contingencias. El porcentaje habitual de contingencia para un proyecto 
de desarrollo de software (o sector tecnológico en general) suele oscilar entre el 10\% y el 20\% 
del presupuesto total. Se ha optado por asignar un \textbf{15\%} del presupuesto total a contingencias, 
lo que nos porporciona los siguientes valores:


\begin{table}[H]
    \begin{center}
    \begin{tabular}{|c|c|c|c|}
        \hline
        \textbf{Recurso} & \textbf{Importe} & \textbf{contingencia} \\ 
        \hline
        Recursos humanos & 12.176,58 € & 1.826,49 € \\
        Recursos materiales directos & 2.676,00 € & 401,40 € \\
        Recursos materiales indirectos  & 6.447,00 € & 967,05 € \\
        \hline
        \multicolumn{2}{|l|}{\textbf{Total}}  & \textbf{3.194,94 €} \\
        \hline
    \end{tabular}
    \caption{Presupuesto total del proyecto. Elaboración propia.}
    \label{tab:calculo_contingencias}
    \end{center}
\end{table}