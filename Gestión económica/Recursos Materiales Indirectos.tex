\subsection{Recursos materiales indirectos}
Continuando con la gestión económica del proyecto, es fundamental no olvidarse 
de los recursos materiales indirectos que, aunque no están directamente vinculados 
a las tareas específicas del proyecto, son esenciales para su desarrollo y éxito. 
Para este proyecto, se han identificado los siguientes recursos materiales indirectos:

\begin{table}[H]
    \begin{center}
    \begin{tabular}{|c|c|c|c|}
        \hline
        \textbf{Nombre del recurso} & \textbf{Coste} & \textbf{Vida útil} & \textbf{Coste Horario}\\ 
        \hline
        Alquiler Oficina (4 meses x 1.278€/mes) & 5.112,00 € & 539 h & 9,484 € \\
        Suministro de Oficina (4 meses x 150€/mes) & 600,00 € & 539 h & 1,113 € \\
        Mobiliario y Ergonomía (3 pers. x 200€) & 600,00 € & 539 h & 1,113 € \\
        Transporte (3 pers. x 45€ trimestral) & 135,00 € & 539 h & 0,250 € \\
        \hline
        \multicolumn{3}{|l|}{\textbf{Total horario} (asumiendo los 3 roles)} & \textbf{11,960 €} \\
        \hline
        \multicolumn{3}{|l|}{\textbf{Total del proyecto} (asumiendo los 3 roles)} & \textbf{6.447,00 €} \\
        \hline
    \end{tabular}
    \caption{Coste horario de los recursos materiales indirectos. Elaboración propia.}
    \label{tab:recursos_indirectos}
    \end{center}
\end{table}

Para calcular el coste del alquiler de la oficina, se ha tomado como referencia el 
precio medio de arrendamiento de oficinas en la zona de Les Corts, Barcelona, 
que fue de 21,30 €/m² en Junio de 2025\cite{properfy}. Dada una superficie 
necesaria de 60 metros cuadrados, el coste mensual total se establece a partir de este valor.