\subsection{Recursos software}
La naturaleza de la organización como entidad sin ánimo de lucro impulsa una estrategia 
rigurosa de optimización de costes, priorizando el uso de soluciones de código abierto 
(\textit{Open Source}) siempre que sea posible. Este enfoque no solo minimiza el gasto 
financiero, sino que también fomenta la transparencia y la adaptabilidad técnica del proyecto.

Para la gestión de recursos de computación y planificación de tareas, se ha optado por 
Slurm, una solución de código abierto ampliamente reconocida en entornos de alto 
rendimiento. En cuanto a la infraestructura colaborativa, la empresa dispone de un acuerdo 
para el uso de Google Workspace, lo que permite gestionar la comunicación y la 
documentación bajo una filosofía de costes reducidos o nulos que se alinea con la estrategia 
\textit{Open Source}.

Para la infraestructura de datos, se utiliza Google Cloud en su plan Estándar, con 
almacenamiento y análisis en BigQuery On-Demand. Según la estimación de cargas del 
proyecto, se prevé un volumen aproximado de:
\begin{itemize}
    \item \textbf{Almacenamiento}: 100--200 GB de datos históricos generados por Slurm, lo que implica un coste mensual aproximado de 4 a 8€ (0,02 €/GB/mes).
    \item \textbf{Consultas}: unos 2 TB procesadas al mes, lo que corresponde a 10 € mensuales (5 €/TB procesado).
\end{itemize}

En total, se estima un coste global de unos \textbf{20–25 € mensuales}, es decir, alrededor de \textbf{80–100 €} para los 4 meses de duración del proyecto. Se ha imputado un presupuesto fijo de \textbf{100 €} para cubrir holguras en picos de almacenamiento o consultas.

Finalmente, la gestión de tareas se realiza a través de \textbf{Jira}, donde se aprovecha una reducción de precio substancial por el estado non-profit de la empresa.

\begin{table}[H]
    \begin{center}
    \begin{tabular}{|c|c|c|c|}
        \hline
        \textbf{Nombre del recurso} & \textbf{Coste} & \textbf{Vida útil} & \textbf{Coste Horario}\\
        \hline
        Google Cloud (BigQuery On-Demand, 4 meses) & 100,00 € & 539 h & 0,186 € \\
        Jira (3 usuarios, coste reducido) & 50,00 € & 539 h & 0,093 € \\
        Slurm (Open Source) & 0,00 € & 539 h & 0,000 € \\
        Google Workspace (Acuerdo non-profit) & 0,00 € & 539 h & 0,000 € \\
        \hline
        \multicolumn{3}{|l|}{\textbf{Total horario} (asumiendo los 3 roles)} & \textbf{0,279 €} \\
        \hline
        \multicolumn{3}{|l|}{\textbf{Total del proyecto} (asumiendo los 3 roles)} & \textbf{150,00 €} \\
        \hline
    \end{tabular}
    \caption{Coste horario de los recursos software indirectos. Elaboración propia.}
    \label{tab:recursos_software}
    \end{center}
\end{table}
