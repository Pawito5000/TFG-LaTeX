\subsection{Imprevistos}

Los imprevistos, a diferencia de las contingencias, son eventos que no se pueden prever ni planificar, pero que pueden afectar al desarrollo del proyecto.
Es por eso que, previamente en la identificación de riesgos, se han identificado ciertos riesgos que pueden llegar a materializarse y afectar al proyecto.
Esto, por supuesto, implicaría un coste añadido al tener que disponer del
personal durante más horas de las prevista

El coste de los imprevistos se calculará de la siguiente manera:
\begin{align}
    \text{Coste del riesgo (€)} =\ & \text{Probabilidad de ocurrencia (\%)} \nonumber \\
    & \times \text{Tiempo de dedicación extra (h)} \nonumber \\
    & \times \text{Sueldo del rol asociado (€)}
\end{align}
En la siguiente tabla se muestran los riesgos que se han considerado:
\begin{table}[h!]
    \centering
    \begin{tabular}{|c|c|c|c|c|}
        \hline
        \textbf{Riesgo} & \textbf{Probabilidad} & \textbf{Dedicación} & \textbf{Rol} & \textbf{Precio}\\
        \hline
        Errores en la transformación de datos & 50\% & 10 h & ID & 175,80 € \\
        Caída de nodos del clúster HPC & 20\% & 5 h & DJ & 18,54 € \\
        Errores en la automatización & 30\% & 10 h &  DJ & 55,62€\\
        \hline
        \multicolumn{4}{|l|}{\textbf{Total}}  & \textbf{249,96 €} \\
        \hline
    \end{tabular}
    \caption{Estimación de costes por imprevistos. Elaboración propia.}
    \label{tab:coste_imprevistos}
\end{table}
