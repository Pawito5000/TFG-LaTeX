\subsection{Recursos materiales directos}
Los gastos de los recursos materiales directos hacen referencia al coste de todo los equipos físicos necesarios para el desarrollo del proyecto.
En este caso, se ha considerado el uso del mismo equipo para los diferentes roles desempeñados, que contiene el material 
necesario para que cada uno pueda realizar sus funciones. Los recursos descritos tienen una vida útil finita, por lo que es necesario
calcular la amortización asociada. Para ello, se ha seguido la siguente fórmula:   
\begin{equation}
    \text{Amortización horaria} = \frac{\text{Valor de adquisición}}{\text{Vida útil en horas}}
\end{equation}


\begin{table}[H]
    \centering
    \begin{tabular}{|c|c|c|c|}
        \hline
        \textbf{Nombre del recurso} & \textbf{Coste} & \textbf{Vida útil} & \textbf{Amortización}\\
        \hline
        hp elitebook 640 14 inch g11 notebook pc & 859,00 € & 539 h & 1,594 € \\
        hp 125 Wired Keyboard PERP & 27,10 € & 539 h & 0,051 € \\
        hp Wired Mouse 1000 & 5,99 € & 539 h &  0,011 €\\
        \hline
        \multicolumn{3}{|l|}{\textbf{Total horario} (asumiendo los 3 equipos)}  & \textbf{4,965 €} \\
        \hline
        \multicolumn{3}{|l|}{\textbf{Total del proyecto} (asumiendo los 3 equipos)}  & \textbf{2.676,00 €} \\
        \hline
    \end{tabular}
    \caption{Amortización de los recursos materiales directos. Elaboración propia.}
    \label{tab:recursos_directos}
\end{table}


En la Tabla 5 se ha considerado la duración total, en horas, del proyecto como la vida útil,
también en horas, de cada material.
