\subsection{Gestión del riesgo}

Una buena planificación temporal debe incluir la gestión de riesgos. En este apartado se identifican
los posibles riesgos que pueden afectar al proyecto, se evalúa su probabilidad e impacto, y se proponen
planes de mitigación para cada uno de ellos.

% Nota: Asegúrate de tener en el preámbulo: \usepackage{tabularx} (y opcionalmente \usepackage{array})
% Si no puedes modificar el preámbulo, sustituye tabularx por tabular y fija manualmente p{...} en lugar de X.
\begin{table}[H]
    \centering
    \small % reduce ligeramente el tamaño de fuente para que quepa
    \setlength{\tabcolsep}{4pt} % reduce separación horizontal entre columnas
    \renewcommand{\arraystretch}{1.2} % mejora legibilidad vertical
    \begin{tabularx}{\textwidth}{|p{2.6cm}|p{2.1cm}|p{1.5cm}|X|}
        \hline
        	\textbf{Riesgo} & \textbf{Probabilidad} & \textbf{Impacto} & \textbf{Plan de mitigación} \\
        \hline
        Errores en la transformación de datos & Media & Medio & Validar y auditar los datos en cada etapa del proceso ETL. Utilizar un entorno de desarrollo para pruebas exhaustivas con datos de muestra previa antes de integrar cambios en el entorno principal. \\
        \hline
        Caída o indisponibilidad de nodos del clúster HPC & Muy alta & Bajo & Planificar el procesamiento en horarios de menor carga. \\
        \hline
        Errores en la automatización & Media & Medio & Realizar pruebas exhaustivas del script de automatización en la máquina virtual. Establecer permisos y configuraciones robustas. Usar un sistema de registro para identificar y depurar errores. \\
        \hline
        Errores típicos de desarrollo & Alta & Bajo & Realizar revisiones de código periódicas. Implementar un control de versiones y realizar copias de seguridad. Utilizar herramientas de depuración y pruebas unitarias. \\
        \hline
    \end{tabularx}
    \caption{Evaluación de riesgos y planes de mitigación. Elaboración propia.}
    \label{tab:riesgos}
\end{table}
