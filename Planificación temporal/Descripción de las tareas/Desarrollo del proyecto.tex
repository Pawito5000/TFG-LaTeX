\subsubsection{Desarrollo del proyecto}
Este bloque engloba las tareas relacionadas con el desarrollo técnico del proyecto. Este bloque
tendría una duración de unas \textbf{351 horas}. Las tareas son:
\begin{itemize}
    \item \textbf{DP-1. Obtención de datos de almacenaje:} Esta tarea consiste en desarrollar 
    los scripts necesarios para extraer los datos de uso, en NetApp y MinIO, de los laboratorios
    y Core Facilities. Estos datos se deben obtener de forma estructurada para su posterior carga 
    en BigQuery. Se estima una dedicación de unas \textbf{30 horas}.
    \newline La tarea no tiene dependencias.
    \newline Recursos: \textit {LaTeX, páginas especializadas de documentación y personal del departamento.}
    
    \item \textbf{DP-2. Modificación de los plugins de Slurm:} Implementación de los \textit{plugins} o modificaciones  en el código fuente de Slurm. 
    El objetivo es recopilar información adicional 
    sobre la heterogeneidad de arquitecturas de CPU y la disponibilidad de nodos \textit{on spot}, lo cual es crucial para el cálculo preciso de costes. 
    Se estima una dedicación de unas \textbf{65 horas}.
    \newline La tarea no tiene dependencias.
    \newline Recursos: \textit {LaTeX, páginas especializadas de documentación y personal del departamento.}
    
    \item \textbf{DP-3. Obtención de datos de cómputo:} Desarrollo del código para extraer los datos de consumo de \textit{jobs} y recursos del clúster HPC 
    a través de Slurm, utilizando la información extendida obtenida en DP-2. Los datos se enviarán a Google BigQuery para su almacenamiento 
    y posterior procesamiento. Se estima una dedicación de unas \textbf{45 horas}.
    \newline La tarea depende de la tarea DP-2.
    \newline Recursos: \textit {LaTeX, páginas especializadas de documentación y personal del departamento.}
    
    \item \textbf{DP-4. Desarrollo del ETL:}  Tarea central que incluye: 1) La lógica de Transformación (T) para calcular la imputación proporcional de costes 
    según el uso de \textit{hardware}, utilizando los datos de DP-1 y DP-3. 2) La Carga (L), que es la integración automática de los resultados en el sistema 
    financiero SAP, utilizando el Service Layer (API de SAP Business One). Se estima una dedicación de unas \textbf{120 horas}.
    \newline La tarea depende de las tareas DP-1 y DP-3.
    \newline Recursos: \textit {LaTeX, páginas especializadas de documentación y personal del departamento.}
    
    \item \textbf{DP-5. Desarrollo del dashboard:}Implementación del \textit{front-end} de visualización. Creación de una interfaz en Open OnDemand 
    que permita a los usuarios (laboratorios y Core Facilities) consultar de manera sencilla sus cuotas de uso y los costes asociados calculados por el proceso ETL. Se estima 
    una dedicación de unas \textbf{50 horas}.
    \newline La tarea depende de la tarea DP-4.
    \newline Recursos: \textit {LaTeX, páginas especializadas de documentación y personal del departamento.}
    
    \item \textbf{DP-6. Pruebas y validación:} Ejecución exhaustiva de pruebas unitarias, de integración y funcionales. Se validará que la recogida de datos sea correcta,
    que la lógica de cálculo de costes sea precisa y que la imputación final en SAP sea correcta. Se incluye la generación y revisión de los \textit{logs} 
    de ejecución para el control de errores. Se estima una dedicación de unas \textbf{25 horas}.
    \newline La tarea depende de la tarea DP-5.
    \newline Recursos: \textit {LaTeX, páginas especializadas de documentación y personal del departamento.}
    
    \item \textbf{DP-7. Despliegue en entorno productivo:} Configuración y puesta en marcha de la solución completa en la máquina virtual Linux, asegurando que el proceso
    se ejecute de manera automática y periódica. Se estima una dedicación de unas \textbf{16 horas}.
    \newline La tarea depende de la tarea DP-6.
    \newline Recursos: \textit {LaTeX, páginas especializadas de documentación y personal del departamento.}
\end{itemize}