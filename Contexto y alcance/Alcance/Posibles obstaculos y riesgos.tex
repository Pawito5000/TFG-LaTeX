\subsection{Posibles obstáculos y riesgos}
Como todo proyecto, y sobre todo al ser de desarrollo, existe una gran posibilidad de verse 
demorado o incluso frustrado por diferentes contratiempos. Por lo tanto, identificarlos es una 
práctica clave para poder actuar en consecuencia. A continuación, se listan las diferentes posibiles situaciones
a ocurrir durante el desarrollo:
\begin{itemize}
    \item Errores en la transformación de datos: durante el proceso ETL (extracción, transformación y carga), 
    podrían producirse inconsistencias o errores de formato que afecten a la exactitud de la imputación de costes.
    \item Complejidad técnica en la modificación de plugins de Slurm: dado que Slurm es un software que no ha 
    sido utilizado previamente por el estudiante, existe el riesgo de que las tareas relacionadas con la modificación 
    y adaptación de este requieran más tiempo del previsto, afectando a la cronología del proyecto.
    \item Caída o indisponibilidad de nodos del clúster HPC: una interrupción en el servicio de alguno de los nodos 
    podría afectar tanto a la recogida de datos como al cálculo de cuotas, generando retrasos en la disponibilidad 
    de la información o resultados incompletos.
    \item Errores en la automatización: la ejecución automática del script en la máquina virtual podría fallar por 
    errores de configuración, permisos insuficientes o actualizaciones del sistema operativo.
\end{itemize}
Para acabar, siempre existe el riesgo de que aparezcan errores típicos asociados al desarrollo de 
proyectos software. Como pueden ser errores en el código, problemas en el despliegue de servicios, entre otros.
