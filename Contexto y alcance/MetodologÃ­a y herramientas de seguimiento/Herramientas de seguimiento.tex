\subsection{Herramientas de seguimiento}

Las herramientas que se utilizarán para el seguimiento del proyecto serán:
\begin{itemize}
    \item \textbf{Jira}: es una herramienta de gestión de proyectos que permite la creación, la asignación y el seguimiento 
    del progreso de las diferentes tareas. Aunque esta herramienta se adapta de manera óptima a la metodología Scrum, en 
    este proyecto se adaptará a un enfoque secuencial para cumplir con los objetivos establecidos.
    \item \textbf{Github}: es una plataforma en la nube que permite el control de versiones \textbf{Git} para alojar y gestionar 
    proyectos de desarrollo de software. Se empleará para asegurar que el código se mantenga organizado y accesible, permitiendo 
    un seguimiento claro de las modificaciones a lo largo del desarrollo.
    \item \textbf{Google Workspace}: es el conjunto de herramientas colaborativas en la nube, ofrecidas por Google. Facilitarán 
    la comunicación y la organización en el proyecto. Entre sus funcionalidades más relevantes destacan: \textbf{Gmail}, que se 
    utilizará para la gestión del correo electrónico y el chat integrado; \textbf{Meet}, que se empleará para la realización de 
    reuniones virtuales; y \textbf{Drive}, que se usará para el almacenamiento y compartición de documentos, con la ventaja de 
    poder acceder desde diferentes dispositivos y cuentas.
\end{itemize}