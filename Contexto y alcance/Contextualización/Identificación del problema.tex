\subsection{Identificación del problema}
Actualmente, la sección de Scientific Computing del departamento de Information Technology Services 
del Instituto de Investigación Biomédica de Barcelona está impulsando un proyecto estratégico orientado 
a la implementación de una nueva infraestructura centralizada de clúster de cálculo de altas prestaciones 
(HPC) y almacenamiento de datos. La finalidad de este cambio es dotar a todo el instituto de una 
infraestructura unificada, más robusta y fácilmente escalable.

Hasta el momento, cada laboratorio y alguna de las Core Facilities\footnote{Una Core Facility es un recurso 
de investigación compartido y centralizado que proporciona a la comunidad científica acceso a instrumentos, 
tecnologías, servicios y expertos únicos y altamente especializados. Los núcleos suelen construirse en torno 
a una tecnología o instrumentación específica, pero no siempre.} del instituto han gestionado y 
financiado de manera independiente sus propios servidores, lo que ha derivado en un sistema heterogéneo 
tanto en las configuraciones técnicas como en los costes asociados. Además, la ausencia de un modelo unificado 
genera una falta de transparencia transversal en la asignación de gastos que afecta de manera directa al departamento 
de Finanzas y las cuentas del Instituto. Por otro lado, los recursos humanos del departamento destinados 
a dar soporte a la infraestructura actual podrían aprovecharse de una forma más eficiente.