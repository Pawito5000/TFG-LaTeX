\subsection{Definición de conceptos}
\begin{itemize}
    \item \textbf{AWS} \newline
    Amazon Web Services (AWS) es una plataforma de servicios en la nube que ofrece una amplia gama de servicios de computación, almacenamiento, bases de datos, entre otros.\cite{QueEsAWS}
    \begin{itemize}
        \item \textbf{AWS Secret Manager}: AWS Secrets Manager ayuda a administrar, recuperar y rotar las credenciales de base de datos, 
        las credenciales de la aplicación, OAuth, los tokens, las claves de la API y otros datos durante todo su ciclo de vida.
    \end{itemize}
   
    \item \textbf{Slurm} \newline
    Slurm es un planificador de cargas de trabajo, denominado \textit{jobs}, de código abierto y altamente escalable, diseñado para 
    clústeres de computación y superordenadores, basados en Linux. Tiene tres funciones principales. Primero asigna recursos 
    de computación (nodos) a los usuarios durante un período de tiempo. Segundo provee de un framework para el lanzamiento, 
    la ejecución y la monitorización de los \textbf{jobs}. Y por último, gestiona como un planificador o \textbf{scheduler} la cola de jobs 
    pendientes y los asigna a los recursos a medida que estos quedan disponibles.\cite{SlurmWorkloadManager}
    
    \item \textbf{Netapp} \newline
    Netapp es una empresa tecnológica que comercializa soluciones de almacenamiento de datos y gestión de la información para entornos híbridos o en la nube. \cite{LiderAlmacenamientoDatos}

    \begin{itemize}
        \item \textbf{ONTAP}: es el sistema operativo desarrollado por Netapp, diseñado para administrar y optimizar el almacenamiento 
        en entornos locales, híbridos y en la nube. Proporciona funcionalidades avanzadas como deduplicación, replicación, snapshots y gestión unificada de datos. 
    \end{itemize}
    
    \item \textbf{MinIO} \newline
    MinIO es una solución de almacenamiento de objetos de alto rendimiento, totalmente compatible con la API de Amazon S3. Está pensada 
    para despliegues en la nube, entornos on-premises o híbridos, y se utiliza para gestionar datos no estructurados como imágenes, vídeos, backups o datasets de machine learning. \cite{S3CompatibleExascale}

    \item \textbf{BigQuery} \newline
    BigQuery es una herramienta de almacenamiento de datos en la nube ofrecida por la división de Google Cloud. Su principal ventaja es que permite a las 
    organizaciones realizar análisis a gran escala de forma rápida y rentable, sin la necesidad de administrar la infraestructura subyacente.

    Además proporciona una manera uniforme de trabajar con datos estructurados y no estructurados. La transmisión de BigQuery admite la transferencia y 
    el análisis continuos de datos, mientras que el motor de análisis distribuido y escalable de BigQuery te permite consultar terabytes en segundos y petabytes en minutos \cite{BigQueryAIData}

    
    \item \textbf{Open OnDemand} \newline
    Open OnDemand es una plataforma web de código abierto que facilita el acceso a los recursos de computación de alto rendimiento (HPC). Sirve como interfaz unificada 
    y basada en el navegador para la gestión y utilización de clústeres y supercomputadores. Su propósito principal es facilitar el acceso a los entornos informáticos, 
    permitiendo a los investigadores, estudiantes y profesionales interactuar con ellos de forma remota a través de una interfaz gráfica de usuario (GUI). Esto elimina la 
    necesidad de dominar la línea de comandos, simplificando la ejecución de trabajos, la monitorización de recursos y la visualización de datos, y por lo tanto, reduce la 
    barrera de entrada para la utilización de HPC. \cite{OpenOnDemandConnecting2025}
\end{itemize}