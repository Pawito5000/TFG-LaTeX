\subsection{Estado del arte}
Gestión de recursos HPC:
\begin{itemize}
    \item Slurm: es una herramienta gratis y open source, orientada a entornos de investigación tales como universidades o institutos de investigación.
    \item LSF: software propiedad de IBM, no es por tanto open source.
\end{itemize}

\begin{table}[H]
    \centering
    \begin{tabular}{|c|c|c|}
        \hline
        \textbf{Herramienta} & \textbf{Licencia} & \textbf{Orientación} \\
        \hline
        Slurm & Open Source & Investigación, universidades \\
        LSF & Propietaria (IBM) & Empresas, investigación \\
        \hline
    \end{tabular}
    \caption{Comparativa de gestores de recursos HPC.}
    \label{tab:gestores_hpc}
\end{table}

Alternativas de ERP\footnote{Un ERP (Enterprise Resource Planning) es un sistema de software que integra y automatiza 
los principales procesos de negocio de una empresa, centralizando datos y flujos de trabajo en una única base de datos 
para mejorar la eficiencia y la toma de decisiones.}:
\begin{itemize}
    \item SAP: plataforma consolidada en la gestión financiera y administrativa de organizaciones de gran tamaño. Dispone de APIs que facilitan la integración con sistemas externos.
    \item Salesforce: orientado principalmente a la gestión de relaciones con clientes (CRM). Aunque permite cierta gestión financiera, no está específicamente diseñado para la imputación de costes internos de HPC.
    \item Oracle ERP\cite{ComoPuedesDotar}: solución robusta y muy extendida en grandes corporaciones, con amplias capacidades de integración, pero con costes elevados y una complejidad de despliegue significativa.
    \item Sage: software de gestión empresarial más orientado a pymes, con funcionalidades contables básicas, pero insuficiente para integrar escenarios complejos como el de HPC.
\end{itemize}

\begin{table}[H]
    \centering
    \begin{tabular}{|c|c|c|}
        \hline
        \textbf{Plataforma} & \textbf{Ventajas} & \textbf{Limitaciones} \\
        \hline
        SAP & Integración, gestión financiera avanzada & Complejidad, coste \\
        Salesforce & CRM, integración básica & No orientado a HPC \\
        Oracle ERP & Robustez, integración & Coste elevado, despliegue complejo \\
        Sage & Simplicidad, orientado a pymes & Funcionalidades limitadas \\
        \hline
    \end{tabular}
    \caption{Comparativa de alternativas ERP.}
    \label{tab:alternativas_erp}
\end{table}

Almacenamiento de datos:
\begin{itemize}
    \item BigQuery: servicio de Google Cloud orientado a análisis de grandes volúmenes de datos. Permite consultas rápidas y escalables, ideal para integrar y transformar datos de uso de HPC y almacenamiento.
    \item Amazon Redshift: alternativa de Amazon Web Services que ofrece capacidades similares de análisis masivo, aunque con una mayor dependencia del ecosistema AWS.
    \item Snowflake: solución de almacenamiento y análisis en la nube que destaca por su flexibilidad multi-cloud, pero que introduce costes de licencia y dependencia tecnológica.
\end{itemize}

\begin{table}[H]
    \centering
    \begin{tabular}{|c|c|c|}
        \hline
        \textbf{Solución} & \textbf{Ventajas} & \textbf{Limitaciones} \\
        \hline
        BigQuery & Consultas rápidas, escalabilidad & Dependencia Google Cloud \\
        Amazon Redshift & Integración AWS, análisis masivo & Dependencia AWS \\
        Snowflake & Multi-cloud, flexibilidad & Coste de licencia, dependencias \\
        \hline
    \end{tabular}
    \caption{Comparativa de soluciones de almacenamiento de datos.}
    \label{tab:almacenamiento_datos}
\end{table}