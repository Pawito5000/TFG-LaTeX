\subsection{Solución escogida}
La elección de Slurm es estratégica, ya que es ampliamente adoptado en centros de investigación, universidades y entornos 
de \textit{scientific computing}\footnote{La Computación Científica o \textit{scientific-computing} es la disciplina que aplica matemáticas, 
informática y otras ciencias para usar ordenadores y desarrollar software que resuelva problemas complejos mediante modelos matemáticos, 
simulaciones y análisis de datos.}. En el caso del IRB Barcelona, como fundación sin ánimo de lucro que depende en gran medida de donaciones, 
resulta clave optimizar los recursos y minimizar los costes asociados a licencias de software, las cuales suelen estar más orientadas 
al ámbito corporativo. Añadir que al ser una herramienta open source, Slurm ofrece la flexibilidad necesaria para ser modificado y adaptado 
a las necesidades específicas de cada proyecto, lo que asegura tanto sostenibilidad como escalabilidad en el tiempo.

En cuanto al ERP, en su momento se optó por SAP Business One, principalmente debido a la disponibilidad del \textit{add-on}\footnote{Un \textit{add-on} es 
un módulo o extensión desarrollado por un tercero que amplía la funcionalidad estándar de un sistema ERP, adaptándolo a requerimientos específicos de una 
organización o sector.} desarrollado por Seidor, el cual cubre de manera eficiente las necesidades particulares de gestión y administración de una institución 
de investigación. Esta elección no es aislada, ya que otras organizaciones del mismo ámbito también han adoptado SAP Business One, lo que refuerza la confianza 
en la robustez de la solución y facilita el intercambio de buenas prácticas y experiencias entre instituciones similares.

Respecto a la infraestructura en la nube, se decidió trabajar con Google Cloud. El principal objetivo fue evitar las limitaciones de escalabilidad
que presentaría una solución basada en una base de datos, MySQL, en local. Especialmente cuando se requiere manejar un gran volumen de datos y consultas 
temporales asociadas. Además, el IRB ya contaba con un contrato activo con Google, lo que, al tratarse de una organización sin ánimo de lucro, permite 
acceder a servicios sin coste adicional.
