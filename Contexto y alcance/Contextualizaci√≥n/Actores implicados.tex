\subsection{Actores implicados}
La implementación de este proyecto supone un cambio organizativo y tecnológico que afecta a diferentes 
actores dentro del instituto. La identificación de dichos actores es fundamental para comprender el alcance, 
los intereses, así como las posibles resistencias y beneficios del proyecto.
\begin{itemize}
    \item El Departamento de Tecnologías de la Información (ITS) - sección de Scientific Computing asume la responsabilidad 
    de la gestión centralizada del clúster y de los servicios de almacenamiento. El objetivo principal es asegurar que los 
    recursos informáticos del instituto sean eficientes, escalables y sostenibles, y que el nuevo modelo de tarificación interna 
    se pueda implementar sin fricciones. 
    \item Los laboratorios de investigación y Core Facilities del IRB son los usuarios finales de esta infraestructura, 
    y por tanto, un actor principal.
    \item El Departamento Financiero del instituto se ve directamente beneficiado por el proyecto. Garantizando que los costes se 
    registran correctamente en el sistema SAP, habiendo una trazabilidad clara de los gastos de cada laboratorio. Haciendo así 
    que el coste del  cálculo del precio a cobrar asociado se vea abismalmente reducido.
    \item El autor del proyecto, Pau Navarro Álvarez.
    \item El equipo de Scientific Computing, que incluye a los administradores de sistemas y otros profesionales de TI.
    \item El resto de personal del proyecto, como el director del proyecto Roberto Riveiro Insua y el tutor Alex Pajuelo Gonzalez.
\end{itemize}