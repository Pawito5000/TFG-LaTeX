\chapter{Results}

\section{Introduction}

This chapter presents the findings of your research. The results are organized according to your research questions or objectives.

\section{Descriptive Analysis}

Provide an overview of your data through descriptive statistics or qualitative descriptions.

\subsection{Sample Characteristics}

If applicable, describe the characteristics of your sample or dataset.

\begin{table}[h]
\centering
\caption{Sample characteristics}
\label{tab:sample_characteristics}
\begin{tabular}{|l|c|}
\hline
\textbf{Characteristic} & \textbf{Value/Frequency} \\
\hline
Total sample size & XXX \\
Age (mean ± SD) & XX.X ± X.X \\
Gender (Male/Female) & XX\% / XX\% \\
Other characteristics & XXX \\
\hline
\end{tabular}
\end{table}

\section{Research Question 1 Results}

Present the results related to your first research question.

\subsection{Main Findings}

Describe the key findings with supporting data, tables, or figures.

\begin{figure}[h]
\centering
% \includegraphics[width=0.8\textwidth]{figures/result1.png}
\caption{Example figure showing key results}
\label{fig:result1}
\end{figure}

\subsection{Statistical Analysis}

If applicable, present statistical test results:

\begin{itemize}
    \item Test statistic: [value]
    \item p-value: [value]
    \item Effect size: [value]
    \item Confidence interval: [range]
\end{itemize}

\section{Research Question 2 Results}

Present the results related to your second research question.

\subsection{Comparative Analysis}

If your research involves comparisons, present them clearly.

\begin{table}[h]
\centering
\caption{Comparison of groups or conditions}
\label{tab:comparison}
\begin{tabular}{|l|c|c|c|}
\hline
\textbf{Variable} & \textbf{Group A} & \textbf{Group B} & \textbf{p-value} \\
\hline
Variable 1 & XX.X ± X.X & XX.X ± X.X & 0.XXX \\
Variable 2 & XX.X ± X.X & XX.X ± X.X & 0.XXX \\
Variable 3 & XX.X ± X.X & XX.X ± X.X & 0.XXX \\
\hline
\end{tabular}
\end{table}

\section{Research Question 3 Results}

Present the results related to your third research question.

\section{Additional Findings}

Present any unexpected or additional findings that emerged from your analysis.

\section{Summary}

Provide a brief summary of the key results, highlighting the most important findings that will be discussed in the following chapter.

The main findings of this study can be summarized as follows:

\begin{enumerate}
    \item Finding 1: [Brief description]
    \item Finding 2: [Brief description]
    \item Finding 3: [Brief description]
\end{enumerate}