\chapter{Additional Data and Analysis}

\section{Introduction}

This appendix contains supplementary material that supports the main text but is too detailed or lengthy to include in the main chapters.

\section{Detailed Statistical Results}

\subsection{Complete Statistical Output}

Include complete statistical output, additional tables, or detailed calculations that support your main results but are too extensive for the main chapters.

\begin{table}[h]
\centering
\caption{Complete correlation matrix}
\label{tab:correlation_matrix}
\begin{tabular}{|l|c|c|c|c|}
\hline
\textbf{Variable} & \textbf{Var1} & \textbf{Var2} & \textbf{Var3} & \textbf{Var4} \\
\hline
Var1 & 1.000 & 0.XXX & 0.XXX & 0.XXX \\
Var2 & 0.XXX & 1.000 & 0.XXX & 0.XXX \\
Var3 & 0.XXX & 0.XXX & 1.000 & 0.XXX \\
Var4 & 0.XXX & 0.XXX & 0.XXX & 1.000 \\
\hline
\end{tabular}
\end{table}

\section{Additional Figures}

Include additional figures that provide supplementary visual information.

\begin{figure}[h]
\centering
% \includegraphics[width=0.8\textwidth]{figures/supplementary_figure.png}
\caption{Supplementary analysis results}
\label{fig:supplementary}
\end{figure}

\section{Research Instruments}

\subsection{Survey Questionnaire}

If applicable, include the complete survey questionnaire or interview protocol used in your research.

\textbf{Example Survey Questions:}

\begin{enumerate}
    \item Demographic information:
    \begin{itemize}
        \item Age: \_\_\_\_
        \item Gender: \_\_\_\_
        \item Education level: \_\_\_\_
    \end{itemize}
    
    \item Research-specific questions:
    \begin{itemize}
        \item Question 1: [Your specific question]
        \item Question 2: [Your specific question]
        \item Question 3: [Your specific question]
    \end{itemize}
\end{enumerate}

\subsection{Interview Protocol}

If you conducted interviews, include your interview guide.

\section{Code Listings}

If your research involves programming or computational work, include relevant code snippets.

\begin{lstlisting}[language=Python, caption=Example data analysis code]
import pandas as pd
import numpy as np
from scipy import stats

# Load data
data = pd.read_csv('data.csv')

# Perform statistical analysis
result = stats.ttest_ind(data['group1'], data['group2'])
print(f"t-statistic: {result.statistic}")
print(f"p-value: {result.pvalue}")
\end{lstlisting}

\section{Additional Documentation}

Include any additional documentation that supports your research, such as:

\begin{itemize}
    \item Ethical approval letters
    \item Participant consent forms (template)
    \item Technical specifications
    \item Detailed methodology descriptions
\end{itemize}