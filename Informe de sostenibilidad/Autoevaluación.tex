\subsection{Autoevaluación}

Una vez respondida la encuesta del proyecto \textit{EDINSOST2-ODS}, cuyo objetivo es evaluar 
el nivel de conocimiento en sostenibilidad de los estudiantes universitarios y sus proyectos, 
se ha procedido a realizar una autoevaluación tanto del proyecto como de mi propio nivel de 
competencia en sostenibilidad.

A nivel personal, me considero una persona consciente del ámbito de la sostenibilidad y trato 
de tenerlo presente en los distintos aspectos que rodean al proyecto, especialmente en las 
dimensiones económica, social y ambiental.

En cuanto a la dimensión ambiental, es el ámbito que mejor conozco y en el que me siento más 
cómodo, ya que es un tema que me interesa y me preocupa profundamente. Considero que, en muchas 
ocasiones, no se le otorga la relevancia que merece.

Respecto a la dimensión social, me resulta más difícil identificar el impacto que un proyecto 
como el mío, centrado en la tarificación de recursos a diferencia de un producto tangible, puede 
generar. No percibo un impacto social directo y, por ello, considero que es el ámbito en el 
que tengo menos desarrollada la competencia.

En la dimensión económica me siento más seguro, ya que entiendo el peso que tiene el proyecto 
en términos de optimización de costes y asignación eficiente de recursos. Sin embargo, esta 
dimensión presenta cierta complejidad adicional debido al elevado impacto económico que puede 
tener su aplicación.

Conozco herramientas y metodologías que pueden ayudar a diseñar o reorientar un proyecto hacia 
un enfoque más sostenible, por lo que considero que podría liderar o transformar un proyecto 
con esta perspectiva.

Por último, en el ámbito ético, aunque no lo he trabajado explícitamente como parte de un 
proyecto, creo que mi formación y contexto social me han proporcionado una base de valores 
éticos y morales sólidos que me permiten tomar decisiones responsables.