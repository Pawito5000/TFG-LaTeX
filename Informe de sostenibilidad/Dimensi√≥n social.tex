\subsection{Dimensión social}

\textbf{¿Qué crees que te va a aportar a nivel personal la realización de este proyecto?}

Este proyecto me va a permitir crecer a nivel personal y profesional a pasos agigantados. 
Sin ir más lejos, me brindará la oportunidad de aplicar los conocimientos teóricos adquiridos 
durante mi formación académica en un contexto real, lo que fortalecerá mis habilidades 
prácticas y mi capacidad para resolver problemas complejos.

\textbf{¿Cómo se resuelve actualmente el problema que quieres abordar(estado del arte)? ¿En 
qué mejorará socialmente tu solución a las existentes?}

Como ya se ha mencionado, el instituto actualmente no cuenta con un sistema de tarificación.
Por lo que a nivel social, los investigadores y estudiantes tendrán una realización
más estrecha con el departamento de informática, lo que fomentará la colaboración y el
intercambio de conocimientos entre ambos, para conseguir un uso más eficiente y responsable 
de los recusrsos.

\textbf{¿Existe una necesidad real del proyecto?}
En una primera instancia, podría parecer que este proyecto no aborda una necesidad social, 
únicamente una motivación económica. Sin embargo, al implementar un sistema de tarificación, 
el instituto puede reaprovechar unos recursos que anteriormente no se estaban utilizando.
Esto abre las puertas a que más investigadores y estudiantes puedan acceder a estos recursos,
acabando en un mayor número de proyectos y trabajos de investigación que, en última instancia,
pueden tener un impacto positivo en la sociedad.