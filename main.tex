\documentclass[a4paper, 11pt, twoside]{article}
\usepackage{wrapfig}
% \usepackage{abstract}
\usepackage{graphicx} % Required for inserting images
\usepackage[a4paper, left=1.15in, right=1.15in, top=1.5in, bottom=1.2in]{geometry}
\usepackage{float}
\usepackage{array}
\usepackage{booktabs}
\usepackage[T1]{fontenc}
\usepackage{amsmath}
\usepackage{tabularx}
\usepackage{graphicx} % Required for inserting images
\usepackage{tikz, xcolor}
\usepackage{hyperref}
\usepackage{cite}
\usepackage{pdfpages}
\usepackage{pdflscape}
\usepackage{afterpage}
\usepackage{textpos}
\usepackage{fancyhdr}
\usepackage{caption}
\usepackage{lmodern,textcomp}
\usepackage[catalan]{babel}
\usepackage{lipsum}
\usepackage{newfloat}
\usepackage{listings}
\usepackage{placeins}
\usepackage[ruled]{algorithm2e}
\usepackage{tcolorbox}
\usepackage[round]{natbib}
\usepackage[toc,page]{appendix}
\usepackage{forest}
\usepackage{pdflscape}
\usepackage{graphicx}
\usepackage{geometry}
\usepackage{quoting,xparse}
\usepackage{svg}



% \NewDocumentCommand{\bywhom}{m}{% the Bourbaki trick
%   {\nobreak\hfill\penalty50\hskip1em\null\nobreak
%    \hfill\mbox{\normalfont(#1)}%
%    \parfillskip=0pt \finalhyphendemerits=0 \par}%
% }

% AIXÒ SERVEIX PER A MODIFICAR UN PROBLEMA AMB ELS CARACTERS --> " + PARAULA 

% \NewDocumentEnvironment{pquotation}{m}
%   {\begin{quoting}[
%      indentfirst=true,
%      leftmargin=\parindent,
%      rightmargin=\parindent]\itshape}
%   {\bywhom{#1}
  
%   \end{quoting}}

% AIXÒ PER INSERIR FRAGMENTS DE CODI

\lstnewenvironment{code}[1][]%
{
   \noindent
   \minipage{\linewidth}
   \vspace{0.5\baselineskip}
   \lstset{
     basicstyle=\ttfamily\footnotesize,
     frame=single,
     breaklines=true,    % Permitir que las líneas se rompan automáticamente
     breakatwhitespace=true, % Permitir la ruptura en los espacios en blanco
     #1
   }
}
{\endminipage}

\lstset{
    language=Java,
    basicstyle=\ttfamily\small,  % Define el tipo de letra y tamaño del código
    keywordstyle=\color{blue},   % Define el color de las palabras clave de Java
    stringstyle=\color{red},     % Define el color de las cadenas de texto
    commentstyle=\color{green},  % Define el color de los comentarios
    morecomment=[s][\color{gray}]{/**}{*/},  % Estilo para comentarios de bloque
    breaklines=true,  % Configura el quiebre automático de línea
    numbers=left,  % Posición de los números de línea
    numberstyle=\tiny\color{gray},  % Estilo para los números de línea
    showspaces=false,  % No mostrar espacios con underscores
    showstringspaces=false,  % No mostrar espacios dentro de strings
    frame=single,  % Agregar un marco alrededor del código
    captionpos=b,  % b = bottom, t = top
}

\definecolor{eclipseStrings}{RGB}{42,0.0,255}
\definecolor{eclipseKeywords}{RGB}{127,0,85}
\colorlet{numb}{magenta!60!black}

\lstdefinelanguage{json}{
    basicstyle=\ttfamily\small,
    showstringspaces=false,
    breaklines=true,
    frame=single,
    literate=
     *{0}{{{\color{numb}0}}}{1}
      {1}{{{\color{numb}1}}}{1}
      {2}{{{\color{numb}2}}}{1}
      {3}{{{\color{numb}3}}}{1}
      {4}{{{\color{numb}4}}}{1}
      {5}{{{\color{numb}5}}}{1}
      {6}{{{\color{numb}6}}}{1}
      {7}{{{\color{numb}7}}}{1}
      {8}{{{\color{numb}8}}}{1}
      {9}{{{\color{numb}9}}}{1}
      {:}{{{\color{punct}{:}}}}{1}
      {,}{{{\color{punct}{,}}}}{1}
      {\{}{{{\color{delim}{\{}}}}{1}
      {\}}{{{\color{delim}{\}}}}}{1}
      {[}{{{\color{delim}{[}}}}{1}
      {]}{{{\color{delim}{]}}}}{1},
    morestring=[b]"
}

\lstdefinelanguage{JavaScript}{
  morekeywords=[1]{break, continue, delete, else, for, function, if, in,
    new, return, this, typeof, var, void, while, with},
  % Literals, primitive types, and reference types.
  morekeywords=[2]{false, null, true, boolean, number, undefined,
    Array, Boolean, Date, Math, Number, String, Object},
  % Built-ins.
  morekeywords=[3]{eval, parseInt, parseFloat, escape, unescape},
  sensitive,
  morecomment=[s]{/*}{*/},
  morecomment=[l]//,
  morecomment=[s]{/**}{*/}, % JavaDoc style comments
  morestring=[b]',
  morestring=[b]"
}[keywords, comments, strings]

\lstdefinelanguage{svelte}{
    morekeywords={async, await, then, if, each, in, on, let, const, function, import, export, from, with, typeof, new, var, return},
    sensitive=true,
    morecomment=[l]{//},
    morecomment=[s]{/*}{*/},
    morecomment=[s]{<!--}{-->},
    morestring=[b]',
    morestring=[b]",
    morestring=[b]`,
    ndkeywords={import, export, return, if, for, while, function, switch, case, break, var, let, const, async, await, then, catch, try, typeof, new, class, extends, implements, interface, constructor, public, private, protected, static, get, set},
    alsoletter={:},
    moredelim=**[is][\color{orange}]{\@\{}{\@\}},
    moredelim=**[is][\color{cyan}]{\$\{}{\}},
    keywordstyle=\color{blue}\bfseries,
    commentstyle=\color{gray}\ttfamily,
    stringstyle=\color{red}\ttfamily,
    identifierstyle=\color{black},
    breaklines=true,
    basicstyle=\ttfamily\small,
    columns=fullflexible,
    showstringspaces=false,
    tabsize=2,
    frame=single,
    captionpos=b,
}


\lstdefinestyle{java}{
    language=Java,
    basicstyle=\ttfamily\small,
    keywordstyle=\color{blue},
    stringstyle=\color{red},
    commentstyle=\color{gray},
    morecomment=[s][\color{gray}]{/**}{*/},
    breaklines=true,
    numbers=left,
    numberstyle=\tiny\color{gray},
    showspaces=false,
    showstringspaces=false,
    frame=single,
    captionpos=b
}

\usepackage{color}
\definecolor{numb}{rgb}{0.58,0,0.82}
\definecolor{punct}{rgb}{0.3,0.3,0.3}
\definecolor{delim}{rgb}{0.5,0.5,0.5}
\definecolor{string}{rgb}{0.6,0.1,0.1}

\lstdefinestyle{json}{
    language=json,
    basicstyle=\ttfamily\small,
    keywordstyle=\color{blue},
    stringstyle=\color{string},
    commentstyle=\color{green},
    morecomment=[s][\color{gray}]{/**}{*/},
    breaklines=true,
    numbers=left,
    numberstyle=\tiny\color{gray},
    showspaces=false,
    showstringspaces=false,
    frame=single,
    captionpos=b
}

\lstdefinelanguage{drl}{
    basicstyle=\ttfamily\small,
    keywordstyle=\color{blue}\bfseries,
    stringstyle=\color{red},
    commentstyle=\color{green},
    morecomment=[s][\color{green}]{/*}{*/},
    morecomment=[l][\color{green}]{//},
    morestring=[b]",
    keywords={rule, when, then, end, salience, eval, update, retract},
    sensitive=true,
    breaklines=true,
    numbers=left,
    numberstyle=\tiny\color{gray},
    showspaces=false,
    showstringspaces=false,
    frame=single,
    captionpos=b
}

\lstdefinestyle{drl}{
    language=drl,
    basicstyle=\ttfamily\small,
    keywordstyle=\color{blue}\bfseries,
    stringstyle=\color{red},
    commentstyle=\color{green},
    breaklines=true,
    numbers=left,
    numberstyle=\tiny\color{gray},
    showspaces=false,
    showstringspaces=false,
    frame=single,
    captionpos=b
}



\renewcommand{\lstlistingname}{Codi}
\renewcommand{\lstlistlistingname}{Índex de Fragments de Codis}
\renewcommand{\appendixpagename}{Annexos}
\renewcommand{\appendixtocname}{Annexos}

\renewcommand*{\listalgorithmcfname}{Llistat de pseudocodi}
\renewcommand*{\algorithmcfname}{Pseudocodi}
\renewcommand*{\algorithmautorefname}{pseudocodi}

\definecolor{darkgreen}{rgb}{0.0, 0.3, 0.0}
\definecolor{lightgreen}{rgb}{0.9, 0.95, 0.9}

\newcommand{\todo}[1]{
    \begin{tcolorbox}[colback=lightgreen, colframe=darkgreen, title=TODO]
        #1
    \end{tcolorbox}
}


% \usepackage{indentfirst} 
% \setlength{\parindent}{1cm}


\usepackage{parskip}
% \setlength{\parindent}{10pt}

% \setlength{\headheight}{15.6pt}
\setlength{\headheight}{16.05263pt}




\vspace{5mm}

\newcolumntype{u}{>{\centering\arraybackslash\hsize=.22\hsize}X}
\newcolumntype{s}{>{\hsize=.45\hsize}X}


%%%%%%%%%%%%%%%%%%%%%%%%%%%%%%%%
% INI creacio subsubsubsecion
%%%%%%%%%%%%%%%%%%%%%%%%%%%%%%%%
\usepackage{titlesec}

\let\oldparagraph\paragraph
\renewcommand{\paragraph}[1]{\oldparagraph{\hspace{30pt}#1}}


\titleclass{\subsubsubsection}{straight}[\subsection]

\newcounter{subsubsubsection}[subsubsection]
\renewcommand\thesubsubsubsection{\thesubsubsection.\arabic{subsubsubsection}}
\renewcommand\theparagraph{\thesubsubsubsection.\arabic{paragraph}} % optional; useful if paragraphs are to be numbered

\titleformat{\subsubsubsection}
  {\normalfont\normalsize\bfseries}{\thesubsubsubsection}{1em}{}
\titlespacing*{\subsubsubsection}
{0pt}{3.25ex plus 1ex minus .2ex}{1.5ex plus .2ex}


\makeatletter
\renewcommand\paragraph{\@startsection{paragraph}{5}{10pt}%
  {3.25ex \@plus1ex \@minus.2ex}%
  {-1em}%
  {\normalfont\normalsize\bfseries}}
\renewcommand\subparagraph{\@startsection{subparagraph}{6}{\parindent}%
  {3.25ex \@plus1ex \@minus .2ex}%
  {-1em}%
  {\normalfont\normalsize\bfseries}}
\def\toclevel@subsubsubsection{4}
\def\toclevel@paragraph{5}
\def\toclevel@paragraph{6}
\def\l@subsubsubsection{\@dottedtocline{4}{7em}{4em}}
\def\l@paragraph{\@dottedtocline{5}{10em}{5em}}
\def\l@subparagraph{\@dottedtocline{6}{14em}{6em}}
\makeatother

\setcounter{secnumdepth}{4}
\setcounter{tocdepth}{3}
%%%%%%%%%%%%%%%%%%%%%%%%%%%%%%%%
% FIN creacio subsubsubtitle
%%%%%%%%%%%%%%%%%%%%%%%%%%%%%%%%


%%%%%%%%%%%%%%%%%%%%%%%%%%%%%%%%
% INI creacio equation
%%%%%%%%%%%%%%%%%%%%%%%%%%%%%%%%
\DeclareFloatingEnvironment[fileext=loeq,placement={!ht},name=Equació]{equationfloat}
%%%%%%%%%%%%%%%%%%%%%%%%%%%%%%%%
% FIN creacio equation
%%%%%%%%%%%%%%%%%%%%%%%%%%%%%%%%

\titleformat{\section}[block]
  {\clearpage\normalfont\Large\bfseries}
  {\thesection}{1em}{}


\renewcommand\refname{Bibliografia}


\linespread{1.25}


\makeatletter
\renewenvironment{abstract}{%
    \if@twocolumn
      \section*{\abstractname}%
    \else 
      \begin{center}%
        {\bfseries \Large\itshape\abstractname\vspace{\z@}}%  % Título en negrita, 14pt y cursiva
      \end{center}%
      \quotation
    \fi}
    {\if@twocolumn\else\endquotation\fi}
\makeatother

% \fancypagestyle{main}{
%     \fancyhf{}
%     \lhead{\small Aplicació Web per a la Visualització i Comparació
% de Trajectòries d’Algorismes d’Optimització}
%     % \rhead{\includegraphics[width=1.2cm]{resources/Icons/logo-fib.png}}
%     \cfoot{\thepage}
%     \setlength{\footskip}{35pt}
% }

\newcommand{\q}[1]{``#1''}

\begin{document}

\hypersetup{
    pdfborder=0 0 0,
}

%% PORTADA I TABLE OF CONTENTS %%

%%%%                                   %%%%
%         INDEX DE LA MEMORIA             %
%%%%                                   %%%%

% \includepdf[pages=-]{portada_fib.pdf}
\input{Extra/GEP}
%\input{Extra/portada}
% \cleardoublepage

% \pagestyle{main}
% \setcounter{page}{11}
\renewcommand{\contentsname}{Continguts}

\tableofcontents
\clearpage
\listoffigures
\begingroup
    \let\clearpage\relax
    \listoftables
\endgroup
\clearpage

\newpage
% \cleardoublepage
%\section{Contextualización}

\subsection{Marco académico}

Este Trabajo de Fin de Grado se enmarca en la Facultad de Informática de Barcelona de la Universidad 
Politécnica de Cataluña \textbf{(UPC)}, como requisito para la obtención del Título de Grado en Ingeniería Informática 
con la especilidad de Tecnologías de la Información. Su desarrollo es crucial para poner en práctica los conocimientos
adquiridos a lo largo de la carrera y demostrar la capacidad para abordar un proyecto de gran envergadura en el ámbito de la informática.

El proyecto se desarrolla en colaboración con el Instituto de Investigación Biomédica de Barcelona \textbf{IRB Barcelona},
un centro de investigación de excelencia dedicado a la investigación biomédica y la formación de científicos en el ambito de la biomedicina.

\subsection{Identificación del problema}
Actualmente, la sección de Scientific Computing del departamento de Information Technology Services 
del Instituto de Investigación Biomédica de Barcelona está impulsando un proyecto estratégico orientado 
a la implementación de una nueva infraestructura centralizada de clúster de cálculo de altas prestaciones 
(HPC) y almacenamiento de datos. La finalidad de este cambio es dotar a todo el instituto de una 
infraestructura unificada, más robusta y fácilmente escalable.

Hasta el momento, cada laboratorio y alguna de las Core Facilities\footnote{Una Core Facility es un recurso 
de investigación compartido y centralizado que proporciona a la comunidad científica acceso a instrumentos, 
tecnologías, servicios y expertos únicos y altamente especializados. Los núcleos suelen construirse en torno 
a una tecnología o instrumentación específica, pero no siempre.} del instituto han gestionado y 
financiado de manera independiente sus propios servidores, lo que ha derivado en un sistema heterogéneo 
tanto en las configuraciones técnicas como en los costes asociados. Además, la ausencia de un modelo unificado 
genera una falta de transparencia transversal en la asignación de gastos que afecta de manera directa al departamento 
de Finanzas y las cuentas del Instituto. Por otro lado, los recursos humanos del departamento destinados 
a dar soporte a la infraestructura actual podrían aprovecharse de una forma más eficiente.

\subsection{Definición de conceptos}
\begin{itemize}
    \item \textbf{AWS} \newline
    Amazon Web Services (AWS) es una plataforma de servicios en la nube que ofrece una amplia gama de servicios de computación, almacenamiento, bases de datos, entre otros.\cite{QueEsAWS}
    \begin{itemize}
        \item \textbf{AWS Secret Manager}: AWS Secrets Manager ayuda a administrar, recuperar y rotar las credenciales de base de datos, 
        las credenciales de la aplicación, OAuth, los tokens, las claves de la API y otros datos durante todo su ciclo de vida.
    \end{itemize}
   
    \item \textbf{Slurm} \newline
    Slurm es un planificador de cargas de trabajo, denominado \textit{jobs}, de código abierto y altamente escalable, diseñado para 
    clústeres de computación y superordenadores, basados en Linux. Tiene tres funciones principales. Primero asigna recursos 
    de computación (nodos) a los usuarios durante un período de tiempo. Segundo provee de un framework para el lanzamiento, 
    la ejecución y la monitorización de los \textbf{jobs}. Y por último, gestiona como un planificador o \textbf{scheduler} la cola de jobs 
    pendientes y los asigna a los recursos a medida que estos quedan disponibles.\cite{SlurmWorkloadManager}
    
    \item \textbf{Netapp} \newline
    Netapp es una empresa tecnológica que comercializa soluciones de almacenamiento de datos y gestión de la información para entornos híbridos o en la nube. \cite{LiderAlmacenamientoDatos}

    \begin{itemize}
        \item \textbf{ONTAP}: es el sistema operativo desarrollado por Netapp, diseñado para administrar y optimizar el almacenamiento 
        en entornos locales, híbridos y en la nube. Proporciona funcionalidades avanzadas como deduplicación, replicación, snapshots y gestión unificada de datos. 
    \end{itemize}
    
    \item \textbf{MinIO} \newline
    MinIO es una solución de almacenamiento de objetos de alto rendimiento, totalmente compatible con la API de Amazon S3. Está pensada 
    para despliegues en la nube, entornos on-premises o híbridos, y se utiliza para gestionar datos no estructurados como imágenes, vídeos, backups o datasets de machine learning. \cite{S3CompatibleExascale}

    \item \textbf{BigQuery} \newline
    BigQuery es una herramienta de almacenamiento de datos en la nube ofrecida por la división de Google Cloud. Su principal ventaja es que permite a las 
    organizaciones realizar análisis a gran escala de forma rápida y rentable, sin la necesidad de administrar la infraestructura subyacente.

    Además proporciona una manera uniforme de trabajar con datos estructurados y no estructurados. La transmisión de BigQuery admite la transferencia y 
    el análisis continuos de datos, mientras que el motor de análisis distribuido y escalable de BigQuery te permite consultar terabytes en segundos y petabytes en minutos \cite{BigQueryAIData}

    
    \item \textbf{Open OnDemand} \newline
    Open OnDemand es una plataforma web de código abierto que facilita el acceso a los recursos de computación de alto rendimiento (HPC). Sirve como interfaz unificada 
    y basada en el navegador para la gestión y utilización de clústeres y supercomputadores. Su propósito principal es facilitar el acceso a los entornos informáticos, 
    permitiendo a los investigadores, estudiantes y profesionales interactuar con ellos de forma remota a través de una interfaz gráfica de usuario (GUI). Esto elimina la 
    necesidad de dominar la línea de comandos, simplificando la ejecución de trabajos, la monitorización de recursos y la visualización de datos, y por lo tanto, reduce la 
    barrera de entrada para la utilización de HPC. \cite{OpenOnDemandConnecting2025}
\end{itemize}

\subsection{Actores implicados}
La implementación de este proyecto supone un cambio organizativo y tecnológico que afecta a diferentes 
actores dentro del instituto. La identificación de dichos actores es fundamental para comprender el alcance, 
los intereses, así como las posibles resistencias y beneficios del proyecto.
\begin{itemize}
    \item El Departamento de Tecnologías de la Información (ITS) - sección de Scientific Computing asume la responsabilidad 
    de la gestión centralizada del clúster y de los servicios de almacenamiento. El objetivo principal es asegurar que los 
    recursos informáticos del instituto sean eficientes, escalables y sostenibles, y que el nuevo modelo de tarificación interna 
    se pueda implementar sin fricciones. 
    \item Los laboratorios de investigación y Core Facilities del IRB son los usuarios finales de esta infraestructura, 
    y por tanto, un actor principal.
    \item El Departamento Financiero del instituto se ve directamente beneficiado por el proyecto. Garantizando que los costes se 
    registran correctamente en el sistema SAP, habiendo una trazabilidad clara de los gastos de cada laboratorio. Haciendo así 
    que el coste del  cálculo del precio a cobrar asociado se vea abismalmente reducido.
    \item El autor del proyecto, Pau Navarro Álvarez.
    \item El equipo de Scientific Computing, que incluye a los administradores de sistemas y otros profesionales de TI.
    \item El resto de personal del proyecto, como el director del proyecto Roberto Riveiro Insua y el tutor Alex Pajuelo Gonzalez.
\end{itemize}

\section{Justificación}
Este proyecto se justifica en la necesidad de sustituir el actual sistema manual de imputación 
de costes de almacenaje y cómputo, basado en hojas de cálculo. Ideando una solución automatizada, 
robusta y escalable. El método actual no solo implica un gran coste humano requiriendo varias semanas 
de trabajo administrativo, sino que también es susceptible a errores humanos y carece de transparencia 
necesaria para un control financiero. En este contexto, resulta conveniente diseñar e implementar una 
solución a medida para la infraestructura y necesidades del instituto, en lugar de utilizar herramientas 
básicas, poco flexibles o incluso genéricas.

\subsection{Estado del arte}
Gestión de recursos HPC:
\begin{itemize}
    \item Slurm: es una herramienta gratis y open source, orientada a entornos de investigación tales como universidades o institutos de investigación.
    \item LSF: software propiedad de IBM, no es por tanto open source.
\end{itemize}

Alternativas de ERP\footnote{Un ERP (Enterprise Resource Planning) es un sistema de software que integra y automatiza 
los principales procesos de negocio de una empresa, centralizando datos y flujos de trabajo en una única base de datos 
para mejorar la eficiencia y la toma de decisiones.}:
\begin{itemize}
    \item SAP: plataforma consolidada en la gestión financiera y administrativa de organizaciones de gran tamaño. Dispone de APIs que facilitan la integración con sistemas externos.
    \item Salesforce: orientado principalmente a la gestión de relaciones con clientes (CRM). Aunque permite cierta gestión financiera, no está específicamente diseñado para la imputación de costes internos de HPC.
    \item Oracle ERP\cite{ComoPuedesDotar}: solución robusta y muy extendida en grandes corporaciones, con amplias capacidades de integración, pero con costes elevados y una complejidad de despliegue significativa.
    \item Sage: software de gestión empresarial más orientado a pymes, con funcionalidades contables básicas, pero insuficiente para integrar escenarios complejos como el de HPC.

\end{itemize}

Almacenamiento de datos:
\begin{itemize}
    \item BigQuery: servicio de Google Cloud orientado a análisis de grandes volúmenes de datos. Permite consultas rápidas y escalables, ideal para integrar y transformar datos de uso de HPC y almacenamiento.
    \item Amazon Redshift: alternativa de Amazon Web Services que ofrece capacidades similares de análisis masivo, aunque con una mayor dependencia del ecosistema AWS.
    \item Snowflake: solución de almacenamiento y análisis en la nube que destaca por su flexibilidad multi-cloud, pero que introduce costes de licencia y dependencia tecnológica.

\end{itemize}

\subsection{Solución escogida}
La elección de Slurm es estratégica, ya que es ampliamente adoptado en centros de investigación, universidades y entornos 
de \textit{scientific computing}\footnote{La Computación Científica o \textit{scientific-computing} es la disciplina que aplica matemáticas, 
informática y otras ciencias para usar ordenadores y desarrollar software que resuelva problemas complejos mediante modelos matemáticos, 
simulaciones y análisis de datos.}. En el caso del IRB Barcelona, como fundación sin ánimo de lucro que depende en gran medida de donaciones, 
resulta clave optimizar los recursos y minimizar los costes asociados a licencias de software, las cuales suelen estar más orientadas 
al ámbito corporativo. Añadir que al ser una herramienta open source, Slurm ofrece la flexibilidad necesaria para ser modificado y adaptado 
a las necesidades específicas de cada proyecto, lo que asegura tanto sostenibilidad como escalabilidad en el tiempo.

En cuanto al ERP, en su momento se optó por SAP Business One, principalmente debido a la disponibilidad del \textit{add-on}\footnote{Un \textit{add-on} es 
un módulo o extensión desarrollado por un tercero que amplía la funcionalidad estándar de un sistema ERP, adaptándolo a requerimientos específicos de una 
organización o sector.} desarrollado por Seidor, el cual cubre de manera eficiente las necesidades particulares de gestión y administración de una institución 
de investigación. Esta elección no es aislada, ya que otras organizaciones del mismo ámbito también han adoptado SAP Business One, lo que refuerza la confianza 
en la robustez de la solución y facilita el intercambio de buenas prácticas y experiencias entre instituciones similares.

Respecto a la infraestructura en la nube, se decidió trabajar con Google Cloud. El principal objetivo fue evitar las limitaciones de escalabilidad
que presentaría una solución basada en una base de datos, MySQL, en local. Especialmente cuando se requiere manejar un gran volumen de datos y consultas 
temporales asociadas. Además, el IRB ya contaba con un contrato activo con Google, lo que, al tratarse de una organización sin ánimo de lucro, permite 
acceder a servicios sin coste adicional.


\section{Alcance}

\subsection{Objetivo principal}
El objetivo principal del proyecto es automatizar la imputación de los costes de computación y 
almacenamiento de los laboratorios del IRB Barcelona en el sistema financiero SAP, mediante el 
desarrollo de una aplicación en Python que procese, calcule e integre la información de consumo de hardware.

\subsection{Objetivos secundarios}
Otros objetivos que deben cumplirse, pero sin ser críticos para el éxito del proyecto son las siguientes:
\begin{itemize}
    \item Desarrollar un front-end (Open OnDemand) que permita a los usuarios consultar de manera sencilla las cuotas de uso y los costes asociados.
    \item Diseñar el sistema para que sea portable, en previsión de la futura migración de los clústeres a un servicio externo de housing\footnote{El servicio de \textit{housing} en un Centro de Procesamiento de Datos (CPD), también llamado \textit{colocation}, consiste en alquilar espacio en el CPD para instalar y operar servidores propios. El proveedor gestiona la infraestructura (electricidad, refrigeración, seguridad y conectividad), mientras el cliente mantiene la propiedad y gestión del hardware.}.
\end{itemize}

\subsection{Requerimientos funcionales}
Los requerimientos funcionales de este proyecto son los siguientes:

\begin{itemize}
    \item El sistema deberá obtener diariamente los datos de consumo de cómputo y almacenamiento de los laboratorios y Core Facilities.
    \item Ampliación del código fuente de Slurm mediante plugins para recopilar información adicional sobre la heterogeneidad de arquitecturas de CPU y la disponibilidad de nodos on spot.
    \item Los datos deberán almacenarse en Google BigQuery como base central de análisis.
    \item La aplicación deberá procesar la información y calcular la imputación proporcional de costes según el uso de hardware y recursos.
    \item El sistema deberá integrar los resultados en el sistema financiero SAP, utilizando el Service Layer (API de SAP Business One).
    \item El proceso deberá ejecutarse de manera automática y periódica en una máquina virtual Linux.
    \item La solución deberá generar registros (logs) de ejecución para garantizar trazabilidad y control de errores.
\end{itemize}


\subsection{Requerimientos no funcionales}
A continuación, se detallan los requerimientos no funcionales para el correcto funcionamiento del proyecto: 
\begin{itemize}
    \item La aplicación deberá estar programada en Python, siguiendo buenas prácticas de desarrollo y con control de versiones en GitHub.
    \item El sistema deberá ser escalable, permitiendo la adaptación a un mayor número de usuarios o recursos en el futuro.
    \item El tiempo de procesamiento diario deberá ser razonable, evitando retrasos en la imputación de costes.
    \item La solución deberá ser segura, garantizando la integridad de los datos transferidos entre BigQuery y SAP.
\end{itemize}


\subsection{Posibles obstáculos y riesgos}
\section{Metodología y herramientas de seguimiento}

\subsection{Metodología}
La metodología que se seguirá durante el desarrollo de este proyecto será de carácter secuencial, 
siguiendo el modelo en cascada. Este enfoque resulta adecuado para el proyecto ya que se desarrollará 
únicamente por una persona, asimismo, los objetivos y alcance están claramente delimitados.

La estructura en cascada permite organizar el trabajo en tareas consecutivas, de manera que cada 
tarea se acaba antes de dar inicio a la siguiente. Así se garantiza una progresión y organización adecuada, 
desde la definición de requisitos hasta la entrega final del proyecto.


\subsection{Herramientas de seguimiento}

Las herramientas que se utilizarán para el seguimiento del proyecto serán:
\begin{itemize}
    \item \textbf{Jira}: es una herramienta de gestión de proyectos que permite la creación, la asignación y el seguimiento 
    del progreso de las diferentes tareas. Aunque esta herramienta se adapta de manera óptima a la metodología Scrum, en 
    este proyecto se adaptará a un enfoque secuencial para cumplir con los objetivos establecidos.
    \item \textbf{Github}: es una plataforma en la nube que permite el control de versiones \textbf{Git} para alojar y gestionar 
    proyectos de desarrollo de software. Se empleará para asegurar que el código se mantenga organizado y accesible, permitiendo 
    un seguimiento claro de las modificaciones a lo largo del desarrollo.
    \item \textbf{Google Workspace}: es el conjunto de herramientas colaborativas en la nube, ofrecidas por Google. Facilitarán 
    la comunicación y la organización en el proyecto. Entre sus funcionalidades más relevantes destacan: \textbf{Gmail}, que se 
    utilizará para la gestión del correo electrónico y el chat integrado; \textbf{Meet}, que se empleará para la realización de 
    reuniones virtuales; y \textbf{Drive}, que se usará para el almacenamiento y compartición de documentos, con la ventaja de 
    poder acceder desde diferentes dispositivos y cuentas.
\end{itemize}

\section{Planificación temporal}
Esta sección del documento detalla de manera exhaustiva la carga de trabajo prevista, 
así como la estimación temporal de cada una de las tareas que componen el proyecto.

La fecha de inicio del proyecto es el 16 de septiembre de 2025 y la fecha de finalización 
prevista es el 20 de enero de 2026, fecha que coincide con el inicio de las defensas de los 
proyectos matriculados en el cuatrimestre de otoño del curso 2025-2026.

Se estima una dedicación diaria de 7 horas de lunes a viernes, lo que supone una dedicación 
semanal de 35 horas. Estas cifras pueden variar, ya que en algunos momentos del proyecto se 
requerirá una mayor dedicación. En total, se espera que la realización completa del proyecto 
abarque unas 595 horas.

\subsection{Descripción de las tareas}
En los siguientes apartados se describen las tareas que componen cada uno de los bloques en los que se divide el proyecto.

\subsection{Gestión del proyecto}\label{ssec:Descripción de las tareas}

\subsubsection{Búsqueda e investigación}\label{ssec:Descripción de las tareas}
Esta sección detalla las tareas asociadas a la búsqueda e investigación necesarias para el desarrollo del proyecto.

\subsubsection{Desarrollo del proyecto}
Este bloque engloba las tareas relacionadas con el desarrollo técnico del proyecto. Este bloque
tendría una duración de unas \textbf{351 horas}. Las tareas son:
\begin{itemize}
    \item \textbf{DP-1. Obtención de datos de almacenaje:} Esta tarea consiste en desarrollar 
    los scripts necesarios para extraer los datos de uso, en NetApp y MinIO, de los laboratorios
    y Core Facilities. Estos datos se deben obtener de forma estructurada para su posterior carga 
    en BigQuery. Se estima una dedicación de unas \textbf{30 horas}.
    \newline La tarea no tiene dependencias.
    \newline Recursos: \textit {LaTeX, páginas especializadas de documentación y personal del departamento.}
    
    \item \textbf{DP-2. Modificación de los plugins de Slurm:} Implementación de los \textit{plugins} o modificaciones  en el código fuente de Slurm. 
    El objetivo es recopilar información adicional 
    sobre la heterogeneidad de arquitecturas de CPU y la disponibilidad de nodos \textit{on spot}, lo cual es crucial para el cálculo preciso de costes. 
    Se estima una dedicación de unas \textbf{65 horas}.
    \newline La tarea no tiene dependencias.
    \newline Recursos: \textit {LaTeX, páginas especializadas de documentación y personal del departamento.}
    
    \item \textbf{DP-3. Obtención de datos de cómputo:} Desarrollo del código para extraer los datos de consumo de \textit{jobs} y recursos del clúster HPC 
    a través de Slurm, utilizando la información extendida obtenida en DP-2. Los datos se enviarán a Google BigQuery para su almacenamiento 
    y posterior procesamiento. Se estima una dedicación de unas \textbf{45 horas}.
    \newline La tarea depende de la tarea DP-2.
    \newline Recursos: \textit {LaTeX, páginas especializadas de documentación y personal del departamento.}
    
    \item \textbf{DP-4. Desarrollo del ETL:}  Tarea central que incluye: 1) La lógica de Transformación (T) para calcular la imputación proporcional de costes 
    según el uso de \textit{hardware}, utilizando los datos de DP-1 y DP-3. 2) La Carga (L), que es la integración automática de los resultados en el sistema 
    financiero SAP, utilizando el Service Layer (API de SAP Business One). Se estima una dedicación de unas \textbf{120 horas}.
    \newline La tarea depende de las tareas DP-1 y DP-3.
    \newline Recursos: \textit {LaTeX, páginas especializadas de documentación y personal del departamento.}
    
    \item \textbf{DP-5. Desarrollo del dashboard:}Implementación del \textit{front-end} de visualización. Creación de una interfaz en Open OnDemand 
    que permita a los usuarios (laboratorios y Core Facilities) consultar de manera sencilla sus cuotas de uso y los costes asociados calculados por el proceso ETL. Se estima 
    una dedicación de unas \textbf{50 horas}.
    \newline La tarea depende de la tarea DP-4.
    \newline Recursos: \textit {LaTeX, páginas especializadas de documentación y personal del departamento.}
    
    \item \textbf{DP-6. Pruebas y validación:} Ejecución exhaustiva de pruebas unitarias, de integración y funcionales. Se validará que la recogida de datos sea correcta,
    que la lógica de cálculo de costes sea precisa y que la imputación final en SAP sea correcta. Se incluye la generación y revisión de los \textit{logs} 
    de ejecución para el control de errores. Se estima una dedicación de unas \textbf{25 horas}.
    \newline La tarea depende de la tarea DP-5.
    \newline Recursos: \textit {LaTeX, páginas especializadas de documentación y personal del departamento.}
    
    \item \textbf{DP-7. Despliegue en entorno productivo:} Configuración y puesta en marcha de la solución completa en la máquina virtual Linux, asegurando que el proceso
    se ejecute de manera automática y periódica. Se estima una dedicación de unas \textbf{16 horas}.
    \newline La tarea depende de la tarea DP-6.
    \newline Recursos: \textit {LaTeX, páginas especializadas de documentación y personal del departamento.}
\end{itemize}

Respecto a los roles de trabajo asociados a cada tarea y sus respectivas descripciones, se pueden observar en la sección de recursos humanos y en tabla \ref{tab:estimaciones_humanos}.
Y cabe destacar que todas las tareas descritas necesitan unos recursos materiales que se detallan en la sección de recursos materiales, tanto directos como indirectos, y en las respectivas tablas \ref{tab:recursos_directos} y \ref{tab:recursos_indirectos}.

\clearpage
\subsection{Estimaciones}
\begin{table}[H]
    \centering
    \begin{tabular}{|c|c|c|c|}
        \hline
        \textbf{ID} & \textbf{Tarea} & \textbf{Tiempo} & \textbf{Dependencias} \\
        \hline
        \textbf{GP} & \textbf{Gestión del Proyecto} & \textbf{138 h} & \textbf{-} \\
        \hline
        GP-1 & Contextualización y alcance & 20 h & - \\
        GP-2 & Planificación temporal & 15 h & GP-1 \\
        GP-3 & Gestión económica y sostenibilidad & 18 h & GP-2 \\
        GP-4 & Entrega final de GEP & 10 h & GP-3 \\
        GP-5 & Reuniones con el director y el departamento & 6 h & - \\
        GP-6 & Redacción de la memoria & 45 h & GP-x\footnotemark, BI-x, DP-x\\
        GP-7 & Presentación de la defensa & 20 h & GP-6 \\
        GP-8 & Defensa del proyecto & 4 h & GP-7 \\
        \hline
        \textbf{BI} & \textbf{Búsqueda e Investigación} & \textbf{50 h} & \textbf{-} \\
        \hline
        BI-1 & Estudio de las soluciones existentes & 15 h & - \\
        BI-2 & Investigación de tecnologías y herramientas & 30 h & - \\
        BI-3 & Estudio de soluciones alternativas & 5 h & - \\
        \hline
        \textbf{DP} & \textbf{Desarrollo del proyecto} & \textbf{351 h} & \textbf{-} \\
        \hline
        DP-1 & Obtención de datos de almacenaje & 30 h & - \\
        DP-2 & Modificación de los plugins de Slurm & 65 h & - \\
        DP-3 & Obtención de datos de cómputo & 45 h & DP-2 \\
        DP-4 & Desarrollo del ETL & 120 h & DP-1 DP-3 \\
        DP-5 & Desarrollo del dashboard & 50 h & DP-4 \\
        DP-6 & Pruebas y validación & 25 h & DP-5 \\
        DP-7 & Despliegue en entorno productivo & 16 h & DP-6 \\
        \hline
    \end{tabular}
    \caption{Resumen del tiempo aproximado para cada tarea}
    \label{tab:estimaciones}
\end{table}

\footnotetext{a excepción de GP-7 y GP-8, porque son tareas posteriores.}\
\documentclass[12pt,twoside,openright]{book}

% Essential packages
\usepackage[utf8]{inputenc}
\usepackage[T1]{fontenc}
\usepackage[english]{babel}
\usepackage{amsmath,amsfonts,amssymb}
\usepackage{graphicx}
\usepackage[margin=2.5cm]{geometry}
\usepackage{setspace}
\usepackage{fancyhdr}
\usepackage{tocbibind}
\usepackage[hidelinks]{hyperref}
\usepackage{caption}
\usepackage{subcaption}
\usepackage{listings}
\usepackage{xcolor}

% Bibliography setup
\usepackage[style=ieee,backend=biber]{biblatex}
\addbibresource{references.bib}

% Code listing setup
\lstset{
    basicstyle=\ttfamily\small,
    keywordstyle=\color{blue},
    stringstyle=\color{red},
    commentstyle=\color{green},
    frame=single,
    breaklines=true,
    showstringspaces=false
}

% Header and footer setup
\pagestyle{fancy}
\fancyhf{}
\fancyhead[LE]{\leftmark}
\fancyhead[RO]{\rightmark}
\fancyfoot[C]{\thepage}
\renewcommand{\headrulewidth}{0.4pt}
\setlength{\headheight}{15pt}

% Line spacing
\onehalfspacing

% Document information
\title{Your Thesis Title Here}
\author{Your Name}
\date{\today}

\begin{document}

% Front matter
\frontmatter
\begin{titlepage}
    \centering
    \vspace*{2cm}
    
    {\LARGE\textbf{University Name}}\\
    \vspace{0.5cm}
    {\large Faculty/Department Name}\\
    \vspace{2cm}
    
    {\Huge\textbf{Your Thesis Title Here}}\\
    \vspace{1cm}
    {\large A thesis submitted in partial fulfillment\\
    of the requirements for the degree of\\
    \textbf{[Degree Title]}}\\
    \vspace{2cm}
    
    {\large\textbf{Author:} Your Name}\\
    \vspace{0.5cm}
    {\large\textbf{Supervisor:} Dr. Supervisor Name}\\
    \vspace{2cm}
    
    {\large\today}
    
    \vfill
\end{titlepage}

\cleardoublepage
\chapter*{Abstract}
\addcontentsline{toc}{chapter}{Abstract}

This is the abstract of your thesis. It should provide a concise summary of your research work, including:

\begin{itemize}
    \item The problem you are addressing
    \item Your methodology or approach
    \item Key findings or results
    \item Main conclusions and implications
\end{itemize}

The abstract should typically be between 150-300 words and provide readers with a clear understanding of what your thesis covers and what you have accomplished.

\textbf{Keywords:} keyword1, keyword2, keyword3, keyword4, keyword5

\cleardoublepage
\chapter*{Acknowledgments}
\addcontentsline{toc}{chapter}{Acknowledgments}

I would like to express my sincere gratitude to all those who contributed to the completion of this thesis.

First and foremost, I thank my supervisor, Dr. [Supervisor Name], for their invaluable guidance, patience, and expertise throughout this research project. Their insights and constructive feedback have been instrumental in shaping this work.

I am also grateful to [University/Institution Name] for providing the necessary resources and facilities that made this research possible.

Special thanks go to my family and friends for their unwavering support and encouragement throughout my academic journey.

Finally, I acknowledge any funding sources, research participants, or collaborators who contributed to this work.

\cleardoublepage
\tableofcontents
\listoffigures
\listoftables

% Main matter
\mainmatter
\chapter{Introduction}

\section{Background and Motivation}

Provide the background context for your research. Explain why this topic is important and what motivated you to pursue this particular area of study.

\section{Problem Statement}

Clearly define the problem or research question that your thesis addresses. This should be specific and well-defined.

\section{Objectives}

List the main objectives of your research:

\begin{enumerate}
    \item Primary objective: [Describe the main goal]
    \item Secondary objective 1: [Describe additional goal]
    \item Secondary objective 2: [Describe additional goal]
\end{enumerate}

\section{Research Questions}

Formulate the specific research questions that guide your investigation:

\begin{itemize}
    \item Research Question 1: [Your question here]
    \item Research Question 2: [Your question here]
    \item Research Question 3: [Your question here]
\end{itemize}

\section{Scope and Limitations}

Define the scope of your research and acknowledge any limitations:

\subsection{Scope}
Describe what your research covers and the boundaries of your investigation.

\subsection{Limitations}
Acknowledge any constraints or limitations in your research approach, data, or methodology.

\section{Thesis Structure}

Provide an overview of how your thesis is organized:

\begin{itemize}
    \item \textbf{Chapter 2}: Literature Review - Reviews existing research in the field
    \item \textbf{Chapter 3}: Methodology - Describes the research approach and methods
    \item \textbf{Chapter 4}: Results - Presents the findings of your research
    \item \textbf{Chapter 5}: Discussion - Analyzes and interprets the results
    \item \textbf{Chapter 6}: Conclusion - Summarizes findings and suggests future work
\end{itemize}
\chapter{Literature Review}

\section{Introduction}

This chapter reviews the existing literature relevant to your research topic. It should demonstrate your understanding of the field and position your work within the broader academic context.

\section{Theoretical Framework}

Describe the theoretical foundations that underpin your research.

\subsection{Key Concepts}

Define and explain the main concepts central to your research.

\subsection{Theoretical Models}

Discuss relevant theoretical models or frameworks that inform your work.

\section{Previous Research}

Review and analyze previous studies in your field.

\subsection{Seminal Works}

Discuss the foundational research that established the field. For example, cite important works like \cite{example2023}.

\subsection{Recent Developments}

Review recent advances and current trends in the research area \cite{example2024}.

\section{Research Gaps}

Identify gaps in the existing literature that your research aims to address.

\begin{itemize}
    \item Gap 1: [Describe the first identified gap]
    \item Gap 2: [Describe the second identified gap]
    \item Gap 3: [Describe the third identified gap]
\end{itemize}

\section{Summary}

Summarize the key points from the literature review and explain how your research will contribute to filling the identified gaps.

This review demonstrates that while significant progress has been made in [your field], there remain important questions that need to be addressed, particularly in the areas of [specific areas]. Your research aims to contribute to this knowledge by [your contribution].
\chapter{Methodology}

\section{Introduction}

This chapter describes the research methodology employed in this study. It outlines the research approach, data collection methods, analysis techniques, and ethical considerations.

\section{Research Approach}

Describe your overall research approach (e.g., quantitative, qualitative, mixed methods).

\subsection{Research Philosophy}

Explain the philosophical stance that guides your research.

\subsection{Research Design}

Detail the specific research design chosen for your study.

\section{Data Collection}

\subsection{Data Sources}

Describe the sources of data for your research:

\begin{itemize}
    \item Primary data sources
    \item Secondary data sources
    \item Any databases or archives used
\end{itemize}

\subsection{Sampling Strategy}

If applicable, describe your sampling approach:

\begin{itemize}
    \item Population definition
    \item Sample size calculation
    \item Sampling method
\end{itemize}

\subsection{Data Collection Instruments}

Describe the tools and instruments used for data collection:

\begin{itemize}
    \item Surveys or questionnaires
    \item Interview protocols
    \item Observation checklists
    \item Measurement instruments
\end{itemize}

\section{Data Analysis}

\subsection{Analytical Framework}

Describe the framework or approach used for analyzing your data.

\subsection{Statistical Methods}

If applicable, list the statistical methods and software used:

\begin{itemize}
    \item Descriptive statistics
    \item Inferential statistical tests
    \item Software packages (e.g., R, SPSS, Python)
\end{itemize}

\subsection{Qualitative Analysis}

If applicable, describe qualitative analysis methods:

\begin{itemize}
    \item Coding procedures
    \item Theme identification
    \item Analysis software
\end{itemize}

\section{Ethical Considerations}

Discuss any ethical issues related to your research and how they were addressed:

\begin{itemize}
    \item Informed consent procedures
    \item Privacy and confidentiality measures
    \item Ethical approval processes
    \item Risk assessment and mitigation
\end{itemize}

\section{Validity and Reliability}

Discuss measures taken to ensure the validity and reliability of your research:

\subsection{Internal Validity}
Describe steps taken to ensure internal validity.

\subsection{External Validity}
Discuss the generalizability of your findings.

\subsection{Reliability}
Explain measures taken to ensure reliability and reproducibility.

\section{Limitations}

Acknowledge any methodological limitations and their potential impact on your findings.
\chapter{Results}

\section{Introduction}

This chapter presents the findings of your research. The results are organized according to your research questions or objectives.

\section{Descriptive Analysis}

Provide an overview of your data through descriptive statistics or qualitative descriptions.

\subsection{Sample Characteristics}

If applicable, describe the characteristics of your sample or dataset.

\begin{table}[h]
\centering
\caption{Sample characteristics}
\label{tab:sample_characteristics}
\begin{tabular}{|l|c|}
\hline
\textbf{Characteristic} & \textbf{Value/Frequency} \\
\hline
Total sample size & XXX \\
Age (mean ± SD) & XX.X ± X.X \\
Gender (Male/Female) & XX\% / XX\% \\
Other characteristics & XXX \\
\hline
\end{tabular}
\end{table}

\section{Research Question 1 Results}

Present the results related to your first research question.

\subsection{Main Findings}

Describe the key findings with supporting data, tables, or figures.

\begin{figure}[h]
\centering
% \includegraphics[width=0.8\textwidth]{figures/result1.png}
\caption{Example figure showing key results}
\label{fig:result1}
\end{figure}

\subsection{Statistical Analysis}

If applicable, present statistical test results:

\begin{itemize}
    \item Test statistic: [value]
    \item p-value: [value]
    \item Effect size: [value]
    \item Confidence interval: [range]
\end{itemize}

\section{Research Question 2 Results}

Present the results related to your second research question.

\subsection{Comparative Analysis}

If your research involves comparisons, present them clearly.

\begin{table}[h]
\centering
\caption{Comparison of groups or conditions}
\label{tab:comparison}
\begin{tabular}{|l|c|c|c|}
\hline
\textbf{Variable} & \textbf{Group A} & \textbf{Group B} & \textbf{p-value} \\
\hline
Variable 1 & XX.X ± X.X & XX.X ± X.X & 0.XXX \\
Variable 2 & XX.X ± X.X & XX.X ± X.X & 0.XXX \\
Variable 3 & XX.X ± X.X & XX.X ± X.X & 0.XXX \\
\hline
\end{tabular}
\end{table}

\section{Research Question 3 Results}

Present the results related to your third research question.

\section{Additional Findings}

Present any unexpected or additional findings that emerged from your analysis.

\section{Summary}

Provide a brief summary of the key results, highlighting the most important findings that will be discussed in the following chapter.

The main findings of this study can be summarized as follows:

\begin{enumerate}
    \item Finding 1: [Brief description]
    \item Finding 2: [Brief description]
    \item Finding 3: [Brief description]
\end{enumerate}
\chapter{Discussion}

\section{Introduction}

This chapter interprets and analyzes the results presented in the previous chapter. It connects the findings to the existing literature and explores their implications.

\section{Interpretation of Results}

\subsection{Research Question 1 Discussion}

Discuss the findings related to your first research question:

\begin{itemize}
    \item What do these results mean?
    \item How do they relate to your hypothesis?
    \item What are the theoretical implications?
\end{itemize}

\subsection{Research Question 2 Discussion}

Discuss the findings related to your second research question:

\begin{itemize}
    \item Compare your results with previous studies
    \item Explain any discrepancies or confirmations
    \item Consider alternative explanations
\end{itemize}

\subsection{Research Question 3 Discussion}

Discuss the findings related to your third research question.

\section{Comparison with Previous Studies}

Compare your findings with those reported in the literature:

\subsection{Confirmatory Findings}

Discuss results that confirm previous research and contribute to the validation of existing theories.

\subsection{Contradictory Findings}

Analyze any results that contradict previous studies and explore possible explanations:

\begin{itemize}
    \item Methodological differences
    \item Population differences
    \item Temporal changes
    \item Contextual factors
\end{itemize}

\subsection{Novel Findings}

Highlight any novel or unexpected findings and discuss their significance.

\section{Theoretical Implications}

Discuss how your findings contribute to theoretical understanding in your field:

\begin{itemize}
    \item Do your results support existing theories?
    \item Do they challenge current paradigms?
    \item What new theoretical insights emerge?
\end{itemize}

\section{Practical Implications}

Discuss the practical applications of your findings:

\subsection{Policy Implications}

If applicable, discuss how your findings might inform policy decisions.

\subsection{Clinical/Professional Practice}

If relevant, discuss implications for professional practice.

\subsection{Industry Applications}

Discuss potential applications in relevant industries or sectors.

\section{Limitations}

Critically examine the limitations of your study and their impact on the interpretation of results:

\subsection{Methodological Limitations}

Discuss limitations related to your research design and methodology.

\subsection{Data Limitations}

Address any limitations related to your data or sample.

\subsection{Analytical Limitations}

Acknowledge limitations in your analytical approach.

\section{Future Research Directions}

Based on your findings and limitations, suggest areas for future research:

\begin{enumerate}
    \item Future research direction 1: [Description and rationale]
    \item Future research direction 2: [Description and rationale]
    \item Future research direction 3: [Description and rationale]
\end{enumerate}

\section{Summary}

Summarize the key points of your discussion and set up the transition to your conclusion chapter.
\chapter{Conclusion}

\section{Introduction}

This final chapter summarizes the key findings of the research, discusses their significance, and outlines the contributions made to the field.

\section{Summary of Findings}

Provide a concise summary of your main findings:

\subsection{Research Question 1 Summary}
Briefly summarize the key findings related to your first research question.

\subsection{Research Question 2 Summary}
Briefly summarize the key findings related to your second research question.

\subsection{Research Question 3 Summary}
Briefly summarize the key findings related to your third research question.

\section{Achievement of Objectives}

Evaluate how well you achieved your research objectives:

\begin{enumerate}
    \item \textbf{Primary objective}: [Describe achievement and evidence]
    \item \textbf{Secondary objective 1}: [Describe achievement and evidence]
    \item \textbf{Secondary objective 2}: [Describe achievement and evidence]
\end{enumerate}

\section{Contributions to Knowledge}

Clearly articulate the contributions your research makes to the field:

\subsection{Theoretical Contributions}

Describe how your work advances theoretical understanding:

\begin{itemize}
    \item New theoretical insights
    \item Validation or refutation of existing theories
    \item Novel conceptual frameworks
\end{itemize}

\subsection{Methodological Contributions}

If applicable, describe methodological innovations or improvements:

\begin{itemize}
    \item New methods or techniques
    \item Improvements to existing approaches
    \item Novel applications of established methods
\end{itemize}

\subsection{Practical Contributions}

Describe practical applications and real-world impact:

\begin{itemize}
    \item Tools or solutions developed
    \item Guidelines or recommendations
    \item Policy implications
\end{itemize}

\section{Implications}

Discuss the broader implications of your work:

\subsection{Academic Implications}

How does your work influence the academic field?

\subsection{Professional Implications}

What are the implications for practitioners in the field?

\subsection{Societal Implications}

If applicable, discuss broader societal impacts.

\section{Limitations and Future Work}

Acknowledge the limitations of your research and suggest future directions:

\subsection{Study Limitations}

Briefly recap the key limitations of your study.

\subsection{Recommendations for Future Research}

Provide specific recommendations for future studies:

\begin{enumerate}
    \item Longitudinal studies to examine [specific aspect]
    \item Replication studies in different contexts
    \item Extension to other populations or settings
    \item Investigation of [specific unexplored aspects]
\end{enumerate}

\section{Final Remarks}

Provide concluding thoughts on your research journey and its significance:

This thesis has [describe overall achievement]. The findings contribute to [field/discipline] by [key contributions]. While limitations exist, the results provide a solid foundation for [future developments]. The research demonstrates [key insight or principle] and opens new avenues for [future research or applications].

Looking forward, this work establishes a foundation for [future developments] and provides valuable insights for [target audience]. The methodological approach developed here can be applied to [broader applications], while the theoretical insights contribute to our understanding of [broader theoretical domain].

% Back matter
\backmatter
\printbibliography

% Appendices
\appendix
\chapter{Additional Data and Analysis}

\section{Introduction}

This appendix contains supplementary material that supports the main text but is too detailed or lengthy to include in the main chapters.

\section{Detailed Statistical Results}

\subsection{Complete Statistical Output}

Include complete statistical output, additional tables, or detailed calculations that support your main results but are too extensive for the main chapters.

\begin{table}[h]
\centering
\caption{Complete correlation matrix}
\label{tab:correlation_matrix}
\begin{tabular}{|l|c|c|c|c|}
\hline
\textbf{Variable} & \textbf{Var1} & \textbf{Var2} & \textbf{Var3} & \textbf{Var4} \\
\hline
Var1 & 1.000 & 0.XXX & 0.XXX & 0.XXX \\
Var2 & 0.XXX & 1.000 & 0.XXX & 0.XXX \\
Var3 & 0.XXX & 0.XXX & 1.000 & 0.XXX \\
Var4 & 0.XXX & 0.XXX & 0.XXX & 1.000 \\
\hline
\end{tabular}
\end{table}

\section{Additional Figures}

Include additional figures that provide supplementary visual information.

\begin{figure}[h]
\centering
% \includegraphics[width=0.8\textwidth]{figures/supplementary_figure.png}
\caption{Supplementary analysis results}
\label{fig:supplementary}
\end{figure}

\section{Research Instruments}

\subsection{Survey Questionnaire}

If applicable, include the complete survey questionnaire or interview protocol used in your research.

\textbf{Example Survey Questions:}

\begin{enumerate}
    \item Demographic information:
    \begin{itemize}
        \item Age: \_\_\_\_
        \item Gender: \_\_\_\_
        \item Education level: \_\_\_\_
    \end{itemize}
    
    \item Research-specific questions:
    \begin{itemize}
        \item Question 1: [Your specific question]
        \item Question 2: [Your specific question]
        \item Question 3: [Your specific question]
    \end{itemize}
\end{enumerate}

\subsection{Interview Protocol}

If you conducted interviews, include your interview guide.

\section{Code Listings}

If your research involves programming or computational work, include relevant code snippets.

\begin{lstlisting}[language=Python, caption=Example data analysis code]
import pandas as pd
import numpy as np
from scipy import stats

# Load data
data = pd.read_csv('data.csv')

# Perform statistical analysis
result = stats.ttest_ind(data['group1'], data['group2'])
print(f"t-statistic: {result.statistic}")
print(f"p-value: {result.pvalue}")
\end{lstlisting}

\section{Additional Documentation}

Include any additional documentation that supports your research, such as:

\begin{itemize}
    \item Ethical approval letters
    \item Participant consent forms (template)
    \item Technical specifications
    \item Detailed methodology descriptions
\end{itemize}

\end{document}
\subsection{Gestión del riesgo}

Una buena planifiacación temporal debe incluir la gestión de riesgos. En este apartado se identifican 
los posibles riesgos que pueden afectar al proyecto, se evalúa su probabilidad e impacto, y se proponen 
planes de mitigación para cada uno de ellos.

\begin{table}[H]
    \centering
    \begin{tabular}{|c|c|c|c|}
        \hline
        \textbf{Rieso} & \textbf{Probabilidad} & \textbf{Impacto} & \textbf{Plan de mitigación} \\
        \hline
        BI-1 & Estudio de las soluciones existentes & 15 h & - \\
        BI-2 & Investigación de tecnologías y herramientas & 30 h & - \\
        BI-3 & Estudio de soluciones alternativas & 5 h & - \\
        
        \hline
    \end{tabular}
    \caption{Tabla de evaluación de riesgos. Elaboración propia.}
    \label{tab:estimaciones}
\end{table}


% \cleardoublepage

% \cleardoublepage


% \documentclass[12pt,twoside,openright]{book}

% Essential packages
\usepackage[utf8]{inputenc}
\usepackage[T1]{fontenc}
\usepackage[english]{babel}
\usepackage{amsmath,amsfonts,amssymb}
\usepackage{graphicx}
\usepackage[margin=2.5cm]{geometry}
\usepackage{setspace}
\usepackage{fancyhdr}
\usepackage{tocbibind}
\usepackage[hidelinks]{hyperref}
\usepackage{caption}
\usepackage{subcaption}
\usepackage{listings}
\usepackage{xcolor}

% Bibliography setup
\usepackage[style=ieee,backend=biber]{biblatex}
\addbibresource{references.bib}

% Code listing setup
\lstset{
    basicstyle=\ttfamily\small,
    keywordstyle=\color{blue},
    stringstyle=\color{red},
    commentstyle=\color{green},
    frame=single,
    breaklines=true,
    showstringspaces=false
}

% Header and footer setup
\pagestyle{fancy}
\fancyhf{}
\fancyhead[LE]{\leftmark}
\fancyhead[RO]{\rightmark}
\fancyfoot[C]{\thepage}
\renewcommand{\headrulewidth}{0.4pt}
\setlength{\headheight}{15pt}

% Line spacing
\onehalfspacing

% Document information
\title{Your Thesis Title Here}
\author{Your Name}
\date{\today}

\begin{document}

% Front matter
\frontmatter
\begin{titlepage}
    \centering
    \vspace*{2cm}
    
    {\LARGE\textbf{University Name}}\\
    \vspace{0.5cm}
    {\large Faculty/Department Name}\\
    \vspace{2cm}
    
    {\Huge\textbf{Your Thesis Title Here}}\\
    \vspace{1cm}
    {\large A thesis submitted in partial fulfillment\\
    of the requirements for the degree of\\
    \textbf{[Degree Title]}}\\
    \vspace{2cm}
    
    {\large\textbf{Author:} Your Name}\\
    \vspace{0.5cm}
    {\large\textbf{Supervisor:} Dr. Supervisor Name}\\
    \vspace{2cm}
    
    {\large\today}
    
    \vfill
\end{titlepage}

\cleardoublepage
\chapter*{Abstract}
\addcontentsline{toc}{chapter}{Abstract}

This is the abstract of your thesis. It should provide a concise summary of your research work, including:

\begin{itemize}
    \item The problem you are addressing
    \item Your methodology or approach
    \item Key findings or results
    \item Main conclusions and implications
\end{itemize}

The abstract should typically be between 150-300 words and provide readers with a clear understanding of what your thesis covers and what you have accomplished.

\textbf{Keywords:} keyword1, keyword2, keyword3, keyword4, keyword5

\cleardoublepage
\chapter*{Acknowledgments}
\addcontentsline{toc}{chapter}{Acknowledgments}

I would like to express my sincere gratitude to all those who contributed to the completion of this thesis.

First and foremost, I thank my supervisor, Dr. [Supervisor Name], for their invaluable guidance, patience, and expertise throughout this research project. Their insights and constructive feedback have been instrumental in shaping this work.

I am also grateful to [University/Institution Name] for providing the necessary resources and facilities that made this research possible.

Special thanks go to my family and friends for their unwavering support and encouragement throughout my academic journey.

Finally, I acknowledge any funding sources, research participants, or collaborators who contributed to this work.

\cleardoublepage
\tableofcontents
\listoffigures
\listoftables

% Main matter
\mainmatter
\chapter{Introduction}

\section{Background and Motivation}

Provide the background context for your research. Explain why this topic is important and what motivated you to pursue this particular area of study.

\section{Problem Statement}

Clearly define the problem or research question that your thesis addresses. This should be specific and well-defined.

\section{Objectives}

List the main objectives of your research:

\begin{enumerate}
    \item Primary objective: [Describe the main goal]
    \item Secondary objective 1: [Describe additional goal]
    \item Secondary objective 2: [Describe additional goal]
\end{enumerate}

\section{Research Questions}

Formulate the specific research questions that guide your investigation:

\begin{itemize}
    \item Research Question 1: [Your question here]
    \item Research Question 2: [Your question here]
    \item Research Question 3: [Your question here]
\end{itemize}

\section{Scope and Limitations}

Define the scope of your research and acknowledge any limitations:

\subsection{Scope}
Describe what your research covers and the boundaries of your investigation.

\subsection{Limitations}
Acknowledge any constraints or limitations in your research approach, data, or methodology.

\section{Thesis Structure}

Provide an overview of how your thesis is organized:

\begin{itemize}
    \item \textbf{Chapter 2}: Literature Review - Reviews existing research in the field
    \item \textbf{Chapter 3}: Methodology - Describes the research approach and methods
    \item \textbf{Chapter 4}: Results - Presents the findings of your research
    \item \textbf{Chapter 5}: Discussion - Analyzes and interprets the results
    \item \textbf{Chapter 6}: Conclusion - Summarizes findings and suggests future work
\end{itemize}
\chapter{Literature Review}

\section{Introduction}

This chapter reviews the existing literature relevant to your research topic. It should demonstrate your understanding of the field and position your work within the broader academic context.

\section{Theoretical Framework}

Describe the theoretical foundations that underpin your research.

\subsection{Key Concepts}

Define and explain the main concepts central to your research.

\subsection{Theoretical Models}

Discuss relevant theoretical models or frameworks that inform your work.

\section{Previous Research}

Review and analyze previous studies in your field.

\subsection{Seminal Works}

Discuss the foundational research that established the field. For example, cite important works like \cite{example2023}.

\subsection{Recent Developments}

Review recent advances and current trends in the research area \cite{example2024}.

\section{Research Gaps}

Identify gaps in the existing literature that your research aims to address.

\begin{itemize}
    \item Gap 1: [Describe the first identified gap]
    \item Gap 2: [Describe the second identified gap]
    \item Gap 3: [Describe the third identified gap]
\end{itemize}

\section{Summary}

Summarize the key points from the literature review and explain how your research will contribute to filling the identified gaps.

This review demonstrates that while significant progress has been made in [your field], there remain important questions that need to be addressed, particularly in the areas of [specific areas]. Your research aims to contribute to this knowledge by [your contribution].
\chapter{Methodology}

\section{Introduction}

This chapter describes the research methodology employed in this study. It outlines the research approach, data collection methods, analysis techniques, and ethical considerations.

\section{Research Approach}

Describe your overall research approach (e.g., quantitative, qualitative, mixed methods).

\subsection{Research Philosophy}

Explain the philosophical stance that guides your research.

\subsection{Research Design}

Detail the specific research design chosen for your study.

\section{Data Collection}

\subsection{Data Sources}

Describe the sources of data for your research:

\begin{itemize}
    \item Primary data sources
    \item Secondary data sources
    \item Any databases or archives used
\end{itemize}

\subsection{Sampling Strategy}

If applicable, describe your sampling approach:

\begin{itemize}
    \item Population definition
    \item Sample size calculation
    \item Sampling method
\end{itemize}

\subsection{Data Collection Instruments}

Describe the tools and instruments used for data collection:

\begin{itemize}
    \item Surveys or questionnaires
    \item Interview protocols
    \item Observation checklists
    \item Measurement instruments
\end{itemize}

\section{Data Analysis}

\subsection{Analytical Framework}

Describe the framework or approach used for analyzing your data.

\subsection{Statistical Methods}

If applicable, list the statistical methods and software used:

\begin{itemize}
    \item Descriptive statistics
    \item Inferential statistical tests
    \item Software packages (e.g., R, SPSS, Python)
\end{itemize}

\subsection{Qualitative Analysis}

If applicable, describe qualitative analysis methods:

\begin{itemize}
    \item Coding procedures
    \item Theme identification
    \item Analysis software
\end{itemize}

\section{Ethical Considerations}

Discuss any ethical issues related to your research and how they were addressed:

\begin{itemize}
    \item Informed consent procedures
    \item Privacy and confidentiality measures
    \item Ethical approval processes
    \item Risk assessment and mitigation
\end{itemize}

\section{Validity and Reliability}

Discuss measures taken to ensure the validity and reliability of your research:

\subsection{Internal Validity}
Describe steps taken to ensure internal validity.

\subsection{External Validity}
Discuss the generalizability of your findings.

\subsection{Reliability}
Explain measures taken to ensure reliability and reproducibility.

\section{Limitations}

Acknowledge any methodological limitations and their potential impact on your findings.
\chapter{Results}

\section{Introduction}

This chapter presents the findings of your research. The results are organized according to your research questions or objectives.

\section{Descriptive Analysis}

Provide an overview of your data through descriptive statistics or qualitative descriptions.

\subsection{Sample Characteristics}

If applicable, describe the characteristics of your sample or dataset.

\begin{table}[h]
\centering
\caption{Sample characteristics}
\label{tab:sample_characteristics}
\begin{tabular}{|l|c|}
\hline
\textbf{Characteristic} & \textbf{Value/Frequency} \\
\hline
Total sample size & XXX \\
Age (mean ± SD) & XX.X ± X.X \\
Gender (Male/Female) & XX\% / XX\% \\
Other characteristics & XXX \\
\hline
\end{tabular}
\end{table}

\section{Research Question 1 Results}

Present the results related to your first research question.

\subsection{Main Findings}

Describe the key findings with supporting data, tables, or figures.

\begin{figure}[h]
\centering
% \includegraphics[width=0.8\textwidth]{figures/result1.png}
\caption{Example figure showing key results}
\label{fig:result1}
\end{figure}

\subsection{Statistical Analysis}

If applicable, present statistical test results:

\begin{itemize}
    \item Test statistic: [value]
    \item p-value: [value]
    \item Effect size: [value]
    \item Confidence interval: [range]
\end{itemize}

\section{Research Question 2 Results}

Present the results related to your second research question.

\subsection{Comparative Analysis}

If your research involves comparisons, present them clearly.

\begin{table}[h]
\centering
\caption{Comparison of groups or conditions}
\label{tab:comparison}
\begin{tabular}{|l|c|c|c|}
\hline
\textbf{Variable} & \textbf{Group A} & \textbf{Group B} & \textbf{p-value} \\
\hline
Variable 1 & XX.X ± X.X & XX.X ± X.X & 0.XXX \\
Variable 2 & XX.X ± X.X & XX.X ± X.X & 0.XXX \\
Variable 3 & XX.X ± X.X & XX.X ± X.X & 0.XXX \\
\hline
\end{tabular}
\end{table}

\section{Research Question 3 Results}

Present the results related to your third research question.

\section{Additional Findings}

Present any unexpected or additional findings that emerged from your analysis.

\section{Summary}

Provide a brief summary of the key results, highlighting the most important findings that will be discussed in the following chapter.

The main findings of this study can be summarized as follows:

\begin{enumerate}
    \item Finding 1: [Brief description]
    \item Finding 2: [Brief description]
    \item Finding 3: [Brief description]
\end{enumerate}
\chapter{Discussion}

\section{Introduction}

This chapter interprets and analyzes the results presented in the previous chapter. It connects the findings to the existing literature and explores their implications.

\section{Interpretation of Results}

\subsection{Research Question 1 Discussion}

Discuss the findings related to your first research question:

\begin{itemize}
    \item What do these results mean?
    \item How do they relate to your hypothesis?
    \item What are the theoretical implications?
\end{itemize}

\subsection{Research Question 2 Discussion}

Discuss the findings related to your second research question:

\begin{itemize}
    \item Compare your results with previous studies
    \item Explain any discrepancies or confirmations
    \item Consider alternative explanations
\end{itemize}

\subsection{Research Question 3 Discussion}

Discuss the findings related to your third research question.

\section{Comparison with Previous Studies}

Compare your findings with those reported in the literature:

\subsection{Confirmatory Findings}

Discuss results that confirm previous research and contribute to the validation of existing theories.

\subsection{Contradictory Findings}

Analyze any results that contradict previous studies and explore possible explanations:

\begin{itemize}
    \item Methodological differences
    \item Population differences
    \item Temporal changes
    \item Contextual factors
\end{itemize}

\subsection{Novel Findings}

Highlight any novel or unexpected findings and discuss their significance.

\section{Theoretical Implications}

Discuss how your findings contribute to theoretical understanding in your field:

\begin{itemize}
    \item Do your results support existing theories?
    \item Do they challenge current paradigms?
    \item What new theoretical insights emerge?
\end{itemize}

\section{Practical Implications}

Discuss the practical applications of your findings:

\subsection{Policy Implications}

If applicable, discuss how your findings might inform policy decisions.

\subsection{Clinical/Professional Practice}

If relevant, discuss implications for professional practice.

\subsection{Industry Applications}

Discuss potential applications in relevant industries or sectors.

\section{Limitations}

Critically examine the limitations of your study and their impact on the interpretation of results:

\subsection{Methodological Limitations}

Discuss limitations related to your research design and methodology.

\subsection{Data Limitations}

Address any limitations related to your data or sample.

\subsection{Analytical Limitations}

Acknowledge limitations in your analytical approach.

\section{Future Research Directions}

Based on your findings and limitations, suggest areas for future research:

\begin{enumerate}
    \item Future research direction 1: [Description and rationale]
    \item Future research direction 2: [Description and rationale]
    \item Future research direction 3: [Description and rationale]
\end{enumerate}

\section{Summary}

Summarize the key points of your discussion and set up the transition to your conclusion chapter.
\chapter{Conclusion}

\section{Introduction}

This final chapter summarizes the key findings of the research, discusses their significance, and outlines the contributions made to the field.

\section{Summary of Findings}

Provide a concise summary of your main findings:

\subsection{Research Question 1 Summary}
Briefly summarize the key findings related to your first research question.

\subsection{Research Question 2 Summary}
Briefly summarize the key findings related to your second research question.

\subsection{Research Question 3 Summary}
Briefly summarize the key findings related to your third research question.

\section{Achievement of Objectives}

Evaluate how well you achieved your research objectives:

\begin{enumerate}
    \item \textbf{Primary objective}: [Describe achievement and evidence]
    \item \textbf{Secondary objective 1}: [Describe achievement and evidence]
    \item \textbf{Secondary objective 2}: [Describe achievement and evidence]
\end{enumerate}

\section{Contributions to Knowledge}

Clearly articulate the contributions your research makes to the field:

\subsection{Theoretical Contributions}

Describe how your work advances theoretical understanding:

\begin{itemize}
    \item New theoretical insights
    \item Validation or refutation of existing theories
    \item Novel conceptual frameworks
\end{itemize}

\subsection{Methodological Contributions}

If applicable, describe methodological innovations or improvements:

\begin{itemize}
    \item New methods or techniques
    \item Improvements to existing approaches
    \item Novel applications of established methods
\end{itemize}

\subsection{Practical Contributions}

Describe practical applications and real-world impact:

\begin{itemize}
    \item Tools or solutions developed
    \item Guidelines or recommendations
    \item Policy implications
\end{itemize}

\section{Implications}

Discuss the broader implications of your work:

\subsection{Academic Implications}

How does your work influence the academic field?

\subsection{Professional Implications}

What are the implications for practitioners in the field?

\subsection{Societal Implications}

If applicable, discuss broader societal impacts.

\section{Limitations and Future Work}

Acknowledge the limitations of your research and suggest future directions:

\subsection{Study Limitations}

Briefly recap the key limitations of your study.

\subsection{Recommendations for Future Research}

Provide specific recommendations for future studies:

\begin{enumerate}
    \item Longitudinal studies to examine [specific aspect]
    \item Replication studies in different contexts
    \item Extension to other populations or settings
    \item Investigation of [specific unexplored aspects]
\end{enumerate}

\section{Final Remarks}

Provide concluding thoughts on your research journey and its significance:

This thesis has [describe overall achievement]. The findings contribute to [field/discipline] by [key contributions]. While limitations exist, the results provide a solid foundation for [future developments]. The research demonstrates [key insight or principle] and opens new avenues for [future research or applications].

Looking forward, this work establishes a foundation for [future developments] and provides valuable insights for [target audience]. The methodological approach developed here can be applied to [broader applications], while the theoretical insights contribute to our understanding of [broader theoretical domain].

% Back matter
\backmatter
\printbibliography

% Appendices
\appendix
\chapter{Additional Data and Analysis}

\section{Introduction}

This appendix contains supplementary material that supports the main text but is too detailed or lengthy to include in the main chapters.

\section{Detailed Statistical Results}

\subsection{Complete Statistical Output}

Include complete statistical output, additional tables, or detailed calculations that support your main results but are too extensive for the main chapters.

\begin{table}[h]
\centering
\caption{Complete correlation matrix}
\label{tab:correlation_matrix}
\begin{tabular}{|l|c|c|c|c|}
\hline
\textbf{Variable} & \textbf{Var1} & \textbf{Var2} & \textbf{Var3} & \textbf{Var4} \\
\hline
Var1 & 1.000 & 0.XXX & 0.XXX & 0.XXX \\
Var2 & 0.XXX & 1.000 & 0.XXX & 0.XXX \\
Var3 & 0.XXX & 0.XXX & 1.000 & 0.XXX \\
Var4 & 0.XXX & 0.XXX & 0.XXX & 1.000 \\
\hline
\end{tabular}
\end{table}

\section{Additional Figures}

Include additional figures that provide supplementary visual information.

\begin{figure}[h]
\centering
% \includegraphics[width=0.8\textwidth]{figures/supplementary_figure.png}
\caption{Supplementary analysis results}
\label{fig:supplementary}
\end{figure}

\section{Research Instruments}

\subsection{Survey Questionnaire}

If applicable, include the complete survey questionnaire or interview protocol used in your research.

\textbf{Example Survey Questions:}

\begin{enumerate}
    \item Demographic information:
    \begin{itemize}
        \item Age: \_\_\_\_
        \item Gender: \_\_\_\_
        \item Education level: \_\_\_\_
    \end{itemize}
    
    \item Research-specific questions:
    \begin{itemize}
        \item Question 1: [Your specific question]
        \item Question 2: [Your specific question]
        \item Question 3: [Your specific question]
    \end{itemize}
\end{enumerate}

\subsection{Interview Protocol}

If you conducted interviews, include your interview guide.

\section{Code Listings}

If your research involves programming or computational work, include relevant code snippets.

\begin{lstlisting}[language=Python, caption=Example data analysis code]
import pandas as pd
import numpy as np
from scipy import stats

# Load data
data = pd.read_csv('data.csv')

# Perform statistical analysis
result = stats.ttest_ind(data['group1'], data['group2'])
print(f"t-statistic: {result.statistic}")
print(f"p-value: {result.pvalue}")
\end{lstlisting}

\section{Additional Documentation}

Include any additional documentation that supports your research, such as:

\begin{itemize}
    \item Ethical approval letters
    \item Participant consent forms (template)
    \item Technical specifications
    \item Detailed methodology descriptions
\end{itemize}

\end{document}


\clearpage
\bibliographystyle{plainnat}
% \bibliography{references}


%%%%                                   %%%%
%         FINAL DE LA MEMORIA             %
%%%%                                   %%%%

\end{document}


%% VERSIÓ INCIAL MEMORIA %%

